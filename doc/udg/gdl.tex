% by Sylwester Arabas <slayoo@igf.fuw.edu.pl>
% this file is a part of the GNU Data Language package
% ----------------------------------------------------
% required LaTeX packages:
% - ifthen
% - stringstrings (\def\@MAXSTRINGSIZE{1000} !!!)
%   http://ctan.org/tex-archive/macros/latex/contrib/stringstrings/
%   
% - hyperref
%   http://ctan.org/tex-archive/macros/latex/contrib/hyperref/
% - makeidx (+ makeindex command-line tool)
% - vmargin
%   http://ctan.org/tex-archive/macros/latex/contrib/vmargin/
% - mwcls (document class)                
%   http://ctan.org/tex-archive/macros/latex/contrib/mwcls/
% - multicol
% - nnfootnote
% ----------------------------------------------------

% mwbk for a more European look & feel :)

%\documentclass[10pt,titleauthor,openany]{mwbk}

\documentclass[10pt,titleauthor,openany]{mwbk}
\SetSectionFormatting{section}{14pt plus5pt minus2pt}{\FormatHangHeading{\large}}{5pt plus3pt}
\usepackage[T1]{fontenc} % otherwise underscores are not "searchable in PDFs"
\usepackage{lmodern} % to get a vector font with the T1 enc

% string manipultation macros (also includes the ifthen package)
\usepackage{stringstrings}

% margins fine-tuning
% usage: \setmarginsrb{ leftmargin }{ topmargin }{ rightmargin }{ bottommargin }
% default: \setmarginsrb{35mm}{20mm}{25mm}{15mm}{12pt}{11mm}{0pt}{11mm}
%                     { headheight }{ headsep }{ footheight }{ footskip }
\usepackage{vmargin}
\setpapersize[landscape]{A4}
\setmarginsrb{20mm}{20mm}{20mm}{5mm}{15mm}{10mm}{0mm}{0mm}
\linespread{1}

% index
\usepackage{makeidx}
\makeindex
\renewenvironment{theindex}{\begin{description}}{\end{description}}

% multi-column page layout
\usepackage{multicol}

% the "for" loop construct for LaTeX
\usepackage{forloop}

\ifx\HCode\Undef
  % draft watermark
  \usepackage[draft]{pdfdraftcopy}
  \draftstring{DRAFT}
  \draftfontsize{220pt}
  \draftangle{35}
\else
\fi

% sans-serif font for a better screen-reading experience
\renewcommand{\familydefault}{\sfdefault}

% no paragraph indentation, bigger paragraph separation
\setlength\parindent{0pt}

% non-numbered footnotes
\usepackage{nnfootnote}

% code listings
\usepackage{listings}

% graphics
\usepackage{graphics}

% definitions of GDL-related macros
\usepackage{gdldoc}

% row-clustering in tabulars
\usepackage{multirow}

% prettier math
\usepackage{amsmath}

% the verbatiminput command
\usepackage{verbatim}

% hyperlinks in the output PDF (and HTML)
\ifx\HCode\Undef
  \usepackage[colorlinks,plainpages=false]{hyperref} % plainpages=false for authorindex package
\else
  \usepackage[tex4ht]{hyperref}
\fi

% bibliography
\usepackage[miniindex]{authorindex} % page references in the bibliography
\def\theaipage{\string\hyperpage{\thepage}}
\usepackage[numbers]{natbib}

%mwcls+natbib fix
\renewcommand*{\bibsection}{\section*{\refname}}

%\aipagetypeorder
\bibliographystyle{copernicus} % e.g. http://www.atmospheric-chemistry-and-physics.net/Copernicus.bst

% not numbering sections (e.g. function names and deeper)
\setcounter{secnumdepth}{0}
\setcounter{tocdepth}{1}

% IDL + trademark sign
\def\IDL{IDL\leavevmode\hbox{$\rm {}^{TM}$}}
\def\PVWAVE{PV-WAVE\leavevmode\hbox{$\rm {}^{TM}$}}

\title{
  \Huge GDL -- GNU Data Language\\
  \Large a free/libre/open-source implementation of IDL/PV-WAVE\textsuperscript{*}\nnfoottext{
    \textsuperscript{*}
    IDL (Interactive Data Language) and PV-WAVE (Precision Visuals - Workstation Analysis and Visualization Environment)\\
    are (were) registered trademarks of EXELIS VIS (ITT VIS; Research Systems, Inc.) and Rogue Wave Software (Visual Numerics; Precision Visuals), respectively
  }
}
\author{
  developed by Marc Schellens and The GDL team\\
  documentation maintained by Sylwester Arabas and Alain Coulais
}
\date{\today}

% putting a list of files (packages, styles, etc) used during compilation
\listfiles

\begin{document}
  \maketitle
  \begin{multicols}{3}{\tableofcontents}
\end{multicols}


  \clearpage
  \twocolumn
  \setlength\parskip{4pt}
 
  \section{About GDL}
  GNU Data Language (GDL) is a free/libre/open source incremental compiler 
  compatible with IDL and to some extent with PV-WAVE.
Together with its library routines it serves as a tool for data analysis 
  and visualization in such disciplines as astronomy, geosciences and
  medical imaging. 

GDL as a language is dynamically-typed, vectorized and has 
  object-oriented programming capabilities. 
GDL library routines handle numerical calculations, data visualisation, 
  signal/image processing, interaction with host OS and data input/output. 
GDL supports several data formats such as netCDF, HDF4, HDF5, GRIB, PNG, TIFF, 
  DICOM, etc. 
Graphical output is handled by X11, PostScript, SVG or z-buffer terminals, 
  the last one allowing output graphics (plots) to be saved in a variety of 
  raster graphics formats.
GDL features integrated debugging facilities. 
GDL has also a Python bridge (Python code can be called from GDL; GDL can be compiled 
  as a Python module). 

Packaged versions of GDL are available for several Linux and BSD flavours as well as Mac~OS~X. 
The source code compiles as well on other UNIX systems, including Solaris.
GDL source code is available for download from Sourceforge.net at:
\url{http://sourceforge.net/projects/gnudatalanguage/}.

Other open-source numerical data analysis tools similar to GDL include:
\begin{itemize}
  \item{GNU Octave: \url{http://www.gnu.org/software/octave/}}
  \item{NCL -- NCAR Command Language: \url{http://www.ncl.ucar.edu/}}
  \item{PDL -- Perl Data Language: \url{http://pdl.perl.org/}}
  \item{R: \url{http://www.r-project.org/}}
  \item{Scilab: \url{http://www.scilab.org/}}
  \item{SciPy: \url{http://www.scipy.org/}}
  \item{Yorick: \url{http://yorick.sourceforge.net/}}
\end{itemize}

  \section{License}
  GDL is a free, libre and open-source software released under the 
  GNU General Public License version 2 \citet{bib_GPL}.
It basicaly means that any GDL user has the freedom to run, copy, distribute,
  study, change and improve GDL.

  \section{Credits}
  % when updating this list remember to update the website!

GDL have been developed by a team of volunteers led by {\bf Marc~Schellens} --
  the project's founder and maintainer.
As of 2011 the core team consists additionally of (in alphabetical order)
  Sylwester~Arabas, Alain~Coulais, Gilles Duvert and Jeol~Gales.

Among many good folks who provided patches and valuable feedback (in alphabetical order) there are:
M\'ed\'eric~Bocquien, %
Justin~Bronn, %
Pierre~Chanial, %
Pedro~Corona~Romero, % dialog_pickfile
Christoph~Fuchs, % CALL_EXTERNAL
Nicolas~Galmiche, %
Greg~Huey, % UNIT kw for SPAWN
Gaurav~Khanna, %
Christopher~Lee, %
Maxime~Lenoir, %
Peter~Messmer, %
Gregory~Marchal, %
Thibaut~Mermet, %
Lea~Noreskal, % src/pro/file_*, ...
Orion~Poplawski, % 
Rene~Preusker, % idl_validname.pro
Mateusz~Turcza, % SEM_*, cygwin port
Joanna~Woo, % (cokhavim) plplot issues
H~Xu, % FILE_WHICH
\ldots

GDL contains snippets of code borrowed from other free and open-source projects credited to:
Deepak~Bandyopadhyay, % gzstream
Sergio~Gelato, % (basic_pro.hpp)
Lutz~Kettner, %gzstream
Craig~B.~Markwardt, %(helpform.pro:;   craigm@lheamail.gsfc.nasa.gov)
Paul~Ricchiazzi, %(findex.pro)
Danny~Smith, %(gsl_fun.cpp)
J.D.~Smith, %(hist_nd.pro)
Richard~Schwartz, %(value_locate.pro)
Paul~Wessel, %(gshhs.cpp)
Bob~Withers, %(base64.hpp)
\ldots

Pre-compiled or pre-configured packages of GDL are available for numerous systems thanks to:
Juan~A.~A\~nel, % Debian
Axel~Beckert, % Debian
Markus~Dittrich, % Gentoo
Takeshi~Enomoto, % Macports
S\'ebastien~Fabbro, %
Orlando~Garcia~Feal, % ArchLinux
Gaurav~Khanna, %
Justin~Lecher, % 
Sebastien~Maret, % Fink
Lea~Noreskal, % Debian
Orion~Poplawski, % Fedora
Marius~Schamschula, % HMUG
G\"urkan~Seng\"un, % Debian
Thierry~Thomas, % FreeBSD
\ldots

%libraries:
GDL is written in C++ using the Terence Parr's ANTLR language-recognition framework. 
Most of the library routines are implemented as interfaces to open-source packages 
  such as GNU Scientific Library, PLPlot, FFTW, ImageMagick, and many many more. 

% IDL/PV-WAVE authors
Last but not least, we would like to acknowledge the designers of IDL and PV-WAVE.

Please do report any missing name on the lists above in the same way
  as any other bug in GDL (see section below).

  \section{Providing fedback}
  Your comments are welcome! Let us know what you use GDL for. Or if you don't, why not. 
Which functionality are you missing/would appreciate most for comming versions.
Please send your bug reports, complaints, suggestions, comments and patches 
  using the trackers or forums available at GDL's project website at SourceForge:
  \url{http://sourceforge.net/projects/gnudatalanguage/}.

  \section{Organization of this document}
  This document is divided into two parts:
\begin{itemize}
  \item{User's guide: intended for users developing programs written in GDL,}
  \item{Developer's guide: intended for those interested in developing or packaging GDL.}
\end{itemize}

Most of GDL functionalities are exemplified with short GDL scripts.
For each such script there are two listings provided: a source
  code listing with line numbers to the left and a log of output below, e.g.:
\gdlcodeexample{helloworld_0}{}{}

All scripts are run by invoking \verb=gdl script.pro= what is equivalent
  to loading the script with the @ operator or typing every line of script
  at the GDL's interactive mode command prompt.

Often the scripts contain lines beginning with a dollar sign ''\$'' which is the GDL
  syntax for executing shell commands, e.g.
\gdlcodeexample{helloworld_1}{}{}

If a script involves creation of a plot, the resultant postscript file is
  displayed below the output listing, e.g.:
\gdlcodeexample{helloworld_2}{}{}

While GDL itself reached a beta status of development, the hereby documentation
  is far from reaching an alpha status -- {\bf help is very welcome!}


  \part{User's guide}

  \chapter{Obtaining, installing, and invoking GDL}
  \section{Requirements and supported environments}
  \section{Availability of pre-compiled packages}
  \section{Compiling GDL from source}
  \subsection{Compiler requirements}
  GNU g++
clang
Intel C++

  \subsection{Autotools}
  \input{chapters/compile-autotools}
  \subsection{Cmake}
  GDL uses CMake for build and test automation. 
Instructions on how to compile GDL are provided in the 
  INSTALL.CMake file that is part of the GDL tarball:
\lstinputlisting[frame=trBL]{../../INSTALL.CMake}

A list of all GDL-specific CMake options is provided below
  along with their default values.
\lstinputlisting[frame=trBL]{cmakeoptlist.tmp}

  \section{Installation layout}
  \section{Command-line options}
  \section{Influential environmental variables}

  \chapter{Language reference}
  %\input{chapters/syntax-intro}
  \section{Syntax basics}
  \gdlfunref{IDL-VALIDNAME}
\gdlfunref{TEMPORARY}

  \section{Datatypes}
  \begin{table*}
  \begin{tabular}{@{}l@{}c@{}@{}c@{}@{}c@{}@{}c@{}@{}c@{}@{}c@{}@{}c@{}@{}c@{}@{}c@{}}
    data type                                              & size      & constants       & min                  & max                     & casting              & array allocation         & index array alloc.     & freeing                                \\ \hline
    \multirow{4}{*}{natural numbers incl. zero (unsigned)} & 8b        & 1b              & 0                    & 255                     & \gdlfunref{byte}     & \gdlfunref{bytarr}       & \gdlfunref{bindgen}    & \multirow{4}{*}{\gdlfunref{temporary}} \\
                                                           & 16b       & 1u              & 0                    & 65535                   & \gdlfunref{uint}     & \gdlfunref{uintarr}      & \gdlfunref{uindgen}    &                                        \\
                                                           & 32b       & 1ul             & 0                    & $4\!\cdot\!10^9$        & \gdlfunref{ulong}    & \gdlfunref{ulonarr}      & \gdlfunref{ulindgen}   &                                        \\
                                                           & 64b       & 1ull            & 0                    & $1,\!8\!\cdot\!10^{19}$ & \gdlfunref{ulong64}  & \gdlfunref{ulon64arr}    & \gdlfunref{ul64indgen} &                                        \\ \hline
    \multirow{3}{*}{integer numbers (signed)}              & 16b       & 1               & -32768               & 32767                   & \gdlfunref{fix}      & \gdlfunref{intarr}       & \gdlfunref{indgen}     & \multirow{3}{*}{\gdlfunref{temporary}} \\
                                                           & 32b       & 1l              & -$2\!\cdot\!10^9$    & $2\!\cdot\!10^9$        & \gdlfunref{long}     & \gdlfunref{lonarr}       & \gdlfunref{lindgen}    &                                        \\
                                                           & 64b       & 1ll             & -$9\!\cdot\!10^{18}$ & $9\!\cdot\!10^{18}$     & \gdlfunref{long64}   & \gdlfunref{long64arr}    & \gdlfunref{l64indgen}  &                                        \\ \hline
    \multirow{2}{*}{real numbers}                          & 32b       & 1.              & -$10^{38}$           & $10^{38}$               & \gdlfunref{float}    & \gdlfunref{fltarr}       & \gdlfunref{findgen}    & \multirow{2}{*}{\gdlfunref{temporary}} \\
                                                           & 64b       & 1d              & -$10^{308}$          & $10^{308}$              & \gdlfunref{double}   & \gdlfunref{dblarr}       & \gdlfunref{dindgen}    &                                        \\ \hline
    \multirow{2}{*}{complex numbers}                       & 64b       & complex(1,0)    & 2x float             & 2x float                & \gdlfunref{complex}  & \gdlfunref{complexarr}   & \gdlfunref{cindgen}    & \multirow{2}{*}{\gdlfunref{temporary}} \\       
                                                           & 128b      & dcomplex(1,0)   & 2x double~           & 2x double~              & \gdlfunref{dcomplex} & \gdlfunref{dcomplexarr}  & \gdlfunref{dcindgen}   &                                        \\ \hline
    character (byte) strings                               & variable  & 'one'           & --                   & --                      & \gdlfunref{string}   & \gdlfunref{strarr}       & --                     & \gdlfunref{temporary}                  \\ \hline
    structures                                             & variable  & \{a:1, b:1\}    & --                   & --                      & --                   & \gdlfunref{replicate}    & --                     & \gdlfunref{temporary}                  \\ \hline
    pointers                                               & n/a       & ptr\_new(1)     & --                   & --                      & --                   & \gdlfunref{ptrarr}       & --                     & \gdlfunref{ptr-free}                   \\ \hline
    objects                                                & n/a       & obj\_new('One') & --                   & --                      & --                   & \gdlfunref{objarr}       & --                     & \gdlfunref{obj-destroy} 
  \end{tabular}
\end{table*}
%\footnote{\tiny compileopt, defint32 powoduje domyslny zapis liczb calk w 32b}  
%\footnote{\tiny ptr_new bez argumentu tworzy wskaznik zerowy (null-pointer)}       
%\footnote{\tiny obj\_new() wywoluje konstruktor -- metod� init() danej klasy, bez argumentu tworzy null-object} 
%\footnote{\tiny kazda z metod akceptuje flag� ,,$\slash$nozero'', przy obecnosci ktorej alokowana pamiec nie jest zerowana}
%\footnote{\tiny make\_array() tworzy tablic� typu zadanego w argumencie-- okresl. typu w czasie wykonania programu}
%\footnote{\tiny delvar() zwalnia pamiec dowolnej zmiennej, ale tylko w programie glownym}                                 
%\footnote{\tiny temporary() zwraca wartosc argumentu -- do uzycia przy ostatnim odwolaniu do danych}
%\footnote{\tiny replicate() moze byc uzyta dla dowolnego typu poza obiektami i wskaznikami} 
%\footnote{\tiny heap\_free() rekursywenie usuwa odniesienia, heap\_gc() zwalnia pamiec do ktorej nie prowadza odnies.}
%\textsuperscript{9}\footnote{\tiny heap\_free() rekursywnie usuwa odniesienia, w obu przypadkach wywolywana jest metoda cleanup()} 
\gdlfunref{ASSOC}

\gdlfunref{BYTE}
\gdlfunref{COMPLEX}, \gdlfunref{DCOMPLEX} (\gdlfunref{CONJ}, \gdlfunref{ATAN}, \gdlfunref{IMAGINARY}, \gdlfunref{REAL-PART}) 
\gdlfunref{DOUBLE}
\gdlfunref{FIX}
\gdlfunref{FLOAT}
\gdlfunref{LONG}
\gdlfunref{LONG64}
\gdlfunref{UINT}
\gdlfunref{ULONG}
\gdlfunref{ULONG64}

\gdlfunref{SIZE}

  \section{Operators}
  \gdlfunref{LOGICAL-AND}
\gdlfunref{LOGICAL-OR}
\gdlfunref{LOGICAL-TRUE}

\gdlfunref{SQRT}

  \section{Flow control structures}
  \subsection{Conditional execution}
\subsubsection{IF%
  \index{IF}%
  \index{THEN}%
  \index{BEGIN!in IF/THEN/ELSE statement}%
  \index{ENDIF}%
  \index{ELSE!in IF/THEN/ELSE statement}%
  \index{BEGIN!in IF/THEN/ELSE statement}%
  \index{ENDELSE}%
}
\gdlcodeexample{if_0}{if,then}{}
\gdlcodeexample{if_1}{if,then,else}{}
contrary to... cannot be used in interactive mode nor in batch scripts, but only within ...
\gdlcodeexample{if_2}{}{if,then,begin,endif}

\subsubsection{CASE%
  \index{CASE}%
  \index{OF!in CASE statement}%
  \index{ELSE!in CASE statement}%
  \index{ENDCASE}%
  \index{BREAK!in CASE statement}%
  \index{BEGIN!in CASE statement}%
  \index{ENDCASE}%
}
\subsubsection{SWITCH%
  \index{SWITCH}%
  \index{OF!in SWITCH statement}%
  \index{ELSE!in SWITCH statement}%
  \index{BEGIN!in SWITCH statement}%
  \index{ENDSWITCH}%
  \index{BREAK!in SWITCH statement}%
}
\subsection{Loops}
\subsubsection{FOR%
  \index{FOR}%
  \index{DO!in FOR statement}%
  \index{BEGIN!in FOR statement}%
  \index{ENDFOR}%
  \index{BREAK!in FOR statement}%
  \index{CONTINUE!in FOR statement}%
}
\subsection{FOREACH%
  \index{FOREACH}%
  \index{DO!in FOREACH statement}%
  \index{ENDFOREACH}%
  \index{BREAK!in FOREACH statement}%
  \index{CONTINUE!in FOREACH statement}%
}
FOREACH statement allows to simplify loop constructs when the array 
  index is not used within the loop:
\gdlcodeexample{foreach_0}{}{foreach,do}
As with index variables in FOR loops, the lifetime of the ''loop variables''
  in FOREACH statements extends beyond the loop execution (see example below).
Both BREAK and CONTINUE statements work in FOREACH in the same way as in other 
  loop constructs:
\gdlcodeexample{foreach_1}{}{foreach,do,begin,continue,break,endforeach}
Loop variables in FOREACH statements contain copies of the array elements
  thus assigning them a value within the loop does not change contents 
  of the array and as a potentially bug-prone situation causes a compiler
  warning (see example above).
\subsubsection{REPEAT%
  \index{REPEAT}%
  \index{BEGIN}%
  \index{ENDREP}%
  \index{UNTIL}%
  \index{BREAK!in REPEAT statement}%
  \index{CONTINUE!in CONTINUE statement}%
}
\subsubsection{WHILE%
  \index{WHILE}%
  \index{DO!in WHILE statement}%
  \index{BEGIN!in WHILE statement}%
  \index{ENDWHILE}%
  \index{BREAK!in WHILE statement}%
  \index{CONTINUE!in WHILE statement}%
}
\subsection{Jumps}
\subsubsection{GOTO\index{GOTO statement}}
Highly deprecated as it usually make the code difficult to read
  and prone to errors.
Anyhow, the syntax is as follows
\gdlcodeexample{goto_0}{goto}{}
As most of the flow control operator described in this section GOTO
  is usable only within a GDL routine -- not within a batch script
  which is equivalent to a series of statements in the interactive mode.

  \section{Variable scoping rules}
  \gdlcodeexample{scope_0}{}{}

  \section{Functions and procedures}
  There may exist a function and a procedure of the same name
(e.g. \gdlfunref{PYTHON} and \gdlproref{PYTHON}, \gdlfunref{CALL-METHOD} and \gdlproref{CALL-METHOD})

  \section{Argument passing}
  \gdlfunref{n-params}
\gdlfunref{keyword-set}
\gdlfunref{arg-present}
\gdlfunref{n-elements}
\gdlfunref{size}

\_EXTRA\index{\_EXTRA}
\_STRICT\_EXTRA\index{\_STRICT\_EXTRA}
\_REF\_EXTRA\index{\_REF\_EXTRA}

when by reference, when by value...

Keyword name abbreviations are allowed if unambiguous, e.g.:
\index{abbreviated keyword names}
\gdlcodeexample{arguments-abbr}{}{}

  \section{Arrays}
  \gdlproref{PRINT} (\gdlproref{TV})
\gdlproref{PM}

\gdlfunref{N-ELEMENTS}
\gdlfunref{SIZE}

\gdlfunref{REFORM}
\gdlfunref{REBIN}
\gdlfunref{REVERSE}
\gdlfunref{ROTATE}
\gdlfunref{TRANSPOSE}

\gdlfunref{SORT}
\gdlfunref{UNIQ}

\gdlfunref{WHERE}
\gdlfunref{ARRAY-INDICES}

\gdlfunref{ARRAY-EQUAL}

\gdlfunref{MAKE-ARRAY}
\gdlfunref{REPLICATE} \gdlproref{REPLICATE-INPLACE}

\gdlfunref{BYTARR}
\gdlfunref{COMPLEXARR}
\gdlfunref{DBLARR}
\gdlfunref{DCOMPLEXARR}
\gdlfunref{FLTARR}
\gdlfunref{INTARR}
\gdlfunref{LON64ARR}
\gdlfunref{LONARR}
\gdlfunref{OBJARR}
\gdlfunref{PTRARR}
\gdlfunref{STRARR}
\gdlfunref{UINTARR}
\gdlfunref{ULON64ARR}
\gdlfunref{ULONARR}

\gdlfunref{BINDGEN}
\gdlfunref{CINDGEN}
\gdlfunref{DCINDGEN}
\gdlfunref{DINDGEN}
\gdlfunref{FINDGEN}
\gdlfunref{INDGEN}
\gdlfunref{L64INDEGEN}
\gdlfunref{LINDEGEN}
\gdlfunref{SINDGEN}
\gdlfunref{UINDGEN}
\gdlfunref{UL64INDGEN}
\gdlfunref{ULINDGEN}

\gdlfunref{IDENTITY}

  \section{Structures}
  \gdlfunref{CREATE-STRUCT}
\gdlfunref{N-TAGS}
\gdlproref{STRUCT-ASSIGN}
\gdlfunref{TAG-NAMES}

  \section{System variables (global)}
  \gdlproref{DEFSYSV} 
(checking if running GDL)

  \section{Heap variables (pointers)}
  \gdlproref{HEAP-GC}
\gdlproref{PTRARR}
\gdlproref{PTR-FREE}
\gdlfunref{PTR-NEW}
\gdlfunref{PTR-VALID}

  \section{The HELP procedure}
  \gdlproref{HELP}

  \section{Object-oriented programming}
  \gdlproref{CALL-METHON}
\gdlfunref{CALL-METHON}
\gdlfunref{OBJARR}

\gdlfunref{OBJ-CLASS}
\gdlproref{OBJ-DESTROY}
\gdlfunref{OBJ-ISA}
\gdlfunref{OBJ-NEW}
\gdlfunref{OBJ-VALID}

  \section{Handling Overflows, Floating Point Special Values}
  \gdlfunref{CHECK-MATH}
\gdlfunref{FINITE}
\gdlfunref{MACHAR}

  \section{Error handling}
  \gdlproref{MESSAGE}
\gdlproref{CATCH}
\gdlproref{ON-ERROR}
\gdlproref{ON-IOERROR}
\gdlproref{EXECUTE}

  \section{Compile options}
  \gdlcodeexample{compile_opt_0}{}{compile_opt,idl2}
\gdlcodeexample{compile_opt_1}{}{compile_opt,hidden}

  \chapter{Interpreter commands and built-in debugging facilities}
  \gdlproref{MESSAGE}
\gdlproref{retall}
\gdlproref{stop}
.COMPILE\index{.COMPILE}
.STEP\index{.STEP}
.CONTINUE\index{.CONTINUE}

\gdlproref{CHECK-MATH}

\gdlproref{JOURNAL}
\gdlproref{RECALL-COMMANDS}

\gdlproref{MEMORY} (\gdlfunref{TEMPORARY})

\gdlproref{RESOLVE-ROUTINE}
\gdlfunref{ROUTINE-INFO}
\gdlfunref{ROUTINE-NAMES}
\gdlfunref{SCOPE-VARFETCH}


  \chapter{Maths}

  \section{Basic Scalar, vector and array operations}
  \gdlfunref{TOTAL}
\gdlfunref{SQRT}
\gdlfunref{REVERSE}
\gdlfunref{SHIFT}
\gdlfunref{MAX}
\gdlfunref{MIN}
\gdlfunref{MEAN}
\gdlfunref{NORM}
\gdlfunref{CONVOL}
\gdlfunref{PRODUCT}
\gdlfunref{CROSSP}
\gdlfunref{DERIV}
\gdlfunref{INVERT}
\gdlfunref{MATRIX-MULTIPLY}
\gdlfunref{TRACE}
\gdlfunref{TRANSPOSE} (\gdlfunref{ROTATE})

\gdlfunref{UNIQ}?

  \section{Basic and special function library}
  \gdlfunref{ABS}
\gdlfunref{CEIL}
\gdlfunref{FLOOR} (aka the Gauss' symbol\index{Gauss symbol})
\gdlfunref{ROUND}

\gdlfunref{ALOG}
\gdlfunref{ALOG10}
\gdlfunref{EXP} (\gdlfunref{GSL-EXP})

\gdlfunref{ACOS}
\gdlfunref{ASIN}
\gdlfunref{ATAN}
\gdlfunref{COS}
\gdlfunref{COSH}
\gdlfunref{SIN}
\gdlfunref{SINH}
\gdlfunref{TAN}
\gdlfunref{TANH}
\gdlfunref{LL-ARC-DISTANCE}

\gdlfunref{BESELI}
\gdlfunref{BESELJ}
\gdlfunref{BESELK}
\gdlfunref{BESELY}

\gdlfunref{ERF} 
\gdlfunref{IMSL-ERF} 
\gdlfunref{ERFC}
\gdlfunref{ERRORF}
\gdlfunref{EXPINT}

\gdlfunref{LAGUERRE}
\gdlfunref{LEGENDRE}

\gdlfunref{SPHER-HARM}

\gdlfunref{GAUSSINT}\index{Gaussian probability function}
\gdlfunref{GAUSS-CVF}
\gdlfunref{GAUSS-PDF}

\gdlfunref{T-PDF}

\gdlfunref{FACTORIAL}
\gdlfunref{GAMMA}
\gdlfunref{BETA}
\gdlfunref{IGAMMA}
\gdlfunref{LNGAMMA}

\gdlfunref{PRIMES}

\gdlfunref{VOIGT}

  \section{Linear algebra}
  \gdlproref{LA-TRIRED}
\gdlproref{LUDC}
\gdlproref{SVDC}

\gdlfunref{IDENTITY}
\gdlfunref{REPLICATE} \gdlproref{REPLICATE-INPLACE}

  \section{Statistics}
  \gdlfunref{CORRELATE}

\gdlfunref{HISTOGRAM}
\gdlfunref{HIST-2D} (implemented using \gdlfunref{HIST-ND})

\gdlfunref{IMSL-BINOMIALCOEF}

\gdlfunref{GAUSSINT}\index{Gaussian probability function}
\gdlfunref{GAUSS-CVF}
\gdlfunref{GAUSS-PDF}

\gdlfunref{T-PDF}

\gdlfunref{KURTOSIS}
\gdlfunref{SKEWNESS}
\gdlfunref{MEAN}
\gdlfunref{MIN}
\gdlfunref{MAX}
\gdlfunref{MEDIAN}
\gdlfunref{MEANABSDEV}
\gdlfunref{MOMENT}
\gdlfunref{STDDEV}
\gdlfunref{VARIANCE}

  \section{Interpolation}
  
\gdlfunref{INTERPOL} (implemented using \gdlfunref{FINDEX})
\gdlfunref{INTERPOLATE}

\gdlfunref{REBIN}

\gdlfunref{DERIV}

\gdlfunref{SPL-INIT}
\gdlfunref{SPL-INTERP}

\gdlfunref{VALUE-LOCATE}

  \section{Polynomials}
  \gdlfunref{IMSL-ZEROPOLY}
\gdlfunref{POLY}

  \section{Geometric calculations}
  \gdlfunref{POLY-AREA}
\gdlfunref{TRIGRID}

  \section{Bitwise operations}
  \gdlfunref{ISHFT}
\gdlproref{BYTEORDER}
\gdlfunref{SWAP-ENDIAN}
\gdlproref{SWAP-ENDIAN-INPLACE}

  \section{Function fitting}
  \citet{bib_MPFIT}

  \section{Fourier analysis}
  \gdlfunref{FFT}
\gdlfunref{DIST}

  \section{Multidimensional root-finding}
  \gdlfunref{BROYDEN}
\gdlfunref{IMSL-ZEROPOLY}
\gdlfunref{NEWTON}

  \section{Random numbers}
  \gdlfunref{RANDOMN}
\gdlfunref{RANDOMU}

  \section{Ordinary differential equations}
  \gdlfunref{RK4}

  \section{Wavelet analysis}
  \gdlfunref{WTN}

  \section{Mathematical and physical constants}
  !PI !DPI
\gdlfunref{IDL-CONSTANT}


  \chapter{Input/output, supported data formats}
  \section{Basics -- accessing files and io streams}
  \gdlproref{PRINT}
\gdlproref{PM}
\gdlproref{GET-KBRD}
\gdlproref{READ}

\gdlproref{BYTEORDER}
\gdlproref{CLOSE}
\gdlproref{EOF}

\gdlproref{READ}
\gdlproref{WRITE}

\gdlproref{READF}
\gdlproref{READS}
\gdlproref{READU}

\gdlproref{GET-LUN}
\gdlproref{FREE-LUN}
\gdlproref{POINT-LUN}
\gdlproref{SKIP-LUN}

\gdlproref{OPENR}
\gdlproref{OPENU}
\gdlproref{OPENW}

  \section{ASCII}
  \gdlproref{PRINTF}
\gdlproref{READF}
\gdlproref{READ-ASCII}

  \section{CSV}
  \input{chapters/io-csv}
  \section{Binary data (raw access)}
  \gdlfunref{READ-BINARY}

\gdlproref{BYTEORDER}
\gdlfunref{SWAP-ENDIAN}
\gdlproref{SWAP-ENDIAN-INPLACE}

  \section{FITS}
  Astron

  \section{netCDF}
  \gdlfunref{NCDF-ATTCOPY}
\gdlproref{NCDF-ATTDEL}
\gdlproref{NCDF-ATTGET}
\gdlfunref{NCDF-ATTINQ}
\gdlfunref{NCDF-ATTNAME}
\gdlproref{NCDF-ATTPUT}
\gdlproref{NCDF-ATTRENAME}
\gdlproref{NCDF-CLOSE}
\gdlproref{NCDF-CONTROL}
\gdlfunref{NCDF-CREATE}
\gdlfunref{NCDF-DIMDEF}
\gdlfunref{NCDF-DIMID}
\gdlproref{NCDF-DIMINQ}
\gdlproref{NCDF-DIMRENAME}
\gdlfunref{NCDF-EXISTS}
\gdlfunref{NCDF-INQUIRE}
\gdlfunref{NCDF-OPEN}
\gdlfunref{NCDF-VARDEF}
\gdlproref{NCDF-VARGET}
\gdlproref{NCDF-VARGET1}
\gdlfunref{NCDF-VARID}
\gdlfunref{NCDF-VARINQ}
\gdlproref{NCDF-VARPUT}
\gdlproref{NCDF-VARRENAME}

  \section{HDF4}
  \gdlproref{HDF-CLOSE}
\gdlfunref{HDF-OPEN}

\gdlproref{HDF-SD-ADDDATA}
\gdlfunref{HDF-SD-ATTRFIND}
\gdlproref{HDF-SD-ATTRINFO}
\gdlfunref{HDF-SD-CREATE}
\gdlproref{HDF-SD-DIMGET}
\gdlfunref{HDF-SD-DIMGETID}
\gdlproref{HDF-SD-END}
\gdlproref{HDF-SD-ENDACCESS}
\gdlproref{HDF-SD-FILEINFO}
\gdlproref{HDF-SD-GETDATA}
\gdlproref{HDF-SD-GETINFO}
\gdlfunref{HDF-SD-NAMETOINDEX}
\gdlfunref{HDF-SD-SELECT}
\gdlfunref{HDF-SD-START}

\gdlfunref{HDF-VD-ATTACH}
\gdlproref{HDF-VD-DETACH}
\gdlfunref{HDF-VD-FIND}
\gdlproref{HDF-VD-GET}
\gdlfunref{HDF-VD-READ}

\gdlfunref{HDF-VG-ATTACH}
\gdlproref{HDF-VG-DETACH}
\gdlfunref{HDF-VG-GETID}
\gdlproref{HDF-VG-GETINFO}
\gdlproref{HDF-VG-GETTRS}

  \section{HDF5}
  \gdlproref{H5A-CLOSE}
\gdlfunref{H5A-GET-NAME}
\gdlfunref{H5A-GET-NUM-ATTRS}
\gdlfunref{H5A-GET-SPACE}
\gdlfunref{H5A-GET-TYPE}
\gdlfunref{H5A-OPEN-IDX}
\gdlfunref{H5A-OPEN-NAME}
\gdlfunref{H5A-READ}
\gdlproref{H5D-CLOSE}
\gdlfunref{H5D-GET-SPACE}
\gdlfunref{H5D-GET-TYPE}
\gdlfunref{H5D-OPEN}
\gdlfunref{H5D-READ}
\gdlproref{H5F-CLOSE}
\gdlfunref{H5F-IS-HDF5}
\gdlfunref{H5F-OPEN}
\gdlproref{H5G-CLOSE}
\gdlfunref{H5G-OPEN}
\gdlproref{H5S-CLOSE}
\gdlfunref{H5S-GET-SIMPLE-EXTENT-DIMS}
\gdlproref{H5T-CLOSE}
\gdlfunref{H5T-GET-SIZE}
\gdlfunref{H5-GET-LIBVERSION}

  \section{raster images (TIFF, PNG, JPEG, \ldots)}
  see chapter in Image Processing

  \section{DICOM}
  \section{GRIB}
  
\gdlfunref{GRIBAPI-CLONE}
\gdlproref{GRIBAPI-CLOSE-FILE}
\gdlfunref{GRIBAPI-COUNT-IN-FILE}
\gdlproref{GRIBAPI-GET}
\gdlproref{GRIBAPI-GET-DATA}
\gdlfunref{GRIBAPI-GET-SIZE}
\gdlfunref{GRIBAPI-NEW-FROM-FILE}
\gdlfunref{GRIBAPI-OPEN-FILE}
\gdlproref{GRIBAPI-RELEASE}

  \section{IDL save files}
  \gdlproref{RESTORE}
\gdlproref{SAVE}


  \chapter{Plotting and mapping}
  \section{2D plots}
  \gdlproref{AXIS}
\gdlproref{CONTOUR}
\gdlproref{OPLOT}
\gdlproref{PLOT}
\gdlproref{PLOTERR}
\gdlproref{PLOTS}
\gdlproref{POLYFILL}
\gdlproref{XYOUTS}

  \section{3D plots}
  \gdlproref{SURFACE}
\gdlproref{PLOTS}

  \section{Plotting raster data}
  \gdlfunref{BYTSCL}
\gdlfunref{TV}
\gdlfunref{TVLCT}
\gdlfunref{TVRD}
\gdlfunref{TVSCL}

  \section{Managing multiple windows}
  \gdlproref{WDELETE}
\gdlproref{WINDOW}
\gdlproref{WSHOW}
\gdlproref{WSET}

  \section{Map projections}
  \gdlproref{MAP-CONTINENTS}
\gdlproref{MAP-PROJ-FORWARD}
\gdlproref{MAP-PROJ-INVERSE}

\gdlfunref{LL-ARC-DISTANCE}

\gdlproref{MAP-CLIP-SET}

  \section{Output terminals}
  \gdlproref{SET-PLOT}
\gdlproref{DEVICE}
\gdlproref{CURSOR}
\gdlproref{ERASE}
\gdlproref{FLUSH}

  \section{Working with colours}
  \gdlcodeexample{colours_0}{decomposed=0}{}
\gdlcodeexample{colours_1}{decomposed=1}{}

  \section{Fonts, symbols and text formatting}
  Harshey fonts \citep{bib_Harshey}

  \section{Misc}
  \gdlfunref{CONVERT-COORD}
\gdlfunref{GET-SCREEN-SIZE}


  \chapter{Interaction with host OS}
  \gdlproref{CD}
\gdlproref{POPD} \gdlproref{PUSHD} \gdlproref{PRINTD}
\gdlproref{EXIT}
\gdlproref{WAIT}

  \section{Executing external commands (via shell or not)}
  \gdlproref{SPAWN} (while \gdlfunref{EXECUTE} ...)

  \section{Filesystem operations}
  \gdlproref{CD}
\gdlfunref{FILE-BASENAME}
\gdlproref{FILE-COPY}
\gdlproref{FILE-DELETE}
\gdlfunref{FILE-DIRNAME}
\gdlfunref{FILE-EXPAND-PATH} (\gdlfunref{EXPAND-PATH})
\gdlfunref{FILE-INFO}
\gdlfunref{FILE-LINES}
\gdlproref{FILE-MKDIR}
\gdlfunref{FILE-SAME}
\gdlfunref{FILE-SEARCH}
\gdlfunref{FILE-TEST}
\gdlfunref{FILE-WHICH}
\gdlfunref{FINDFILE}
\gdlfunref{FSTAT}
\gdlfunref{PATH-SEP}

  \section{Network operations}
  \gdlproref{SOCKET}
\gdlfunref{PARSE-URL}

  \section{Command-line options and environmental variables}
  \gdlfunref{COMMAND-LINE-ARGS}
\gdlproref{SETENV}
\gdlfunref{GETENV}
\gdlfunref{LOCALE-GET}


  \chapter{Manipulating strings}
  \gdlfunref{strcmp}
\gdlfunref{strcompress}
\gdlfunref{stregex}
\gdlfunref{strjoin}
\gdlfunref{strlen}

\gdlfunref{strlowercase}
\gdlfunref{strupcase}

\gdlfunref{strmid}
\gdlfunref{strpos}
\gdlfunref{strput}
\gdlfunref{strsplit}
\gdlfunref{strtok}
\gdlfunref{strtrim}
\gdlfunref{str-sep}

\gdlfunref{strarr}
\gdlfunref{string}

% ref special formatting for plots


  \chapter{Representing date \& time}
  \gdlproref{CALDAT}
\gdlproref{CALENDAR}
\gdlfunref{SYSTIME}


  \chapter{Image processing}
  \gdlfunref{QUERY-BMP}
\gdlfunref{QUERY-DICOM}
\gdlfunref{QUERY-GIF}
\gdlfunref{QUERY-IMAGE}
\gdlfunref{QUERY-JPEG}
\gdlfunref{QUERY-PICT}
\gdlfunref{QUERY-PNG}
\gdlfunref{QUERY-PPM}
\gdlfunref{QUERY-TIFF}

\gdlfunref{READ-BMP}
\gdlfunref{READ-DICOM}
\gdlproref{READ-JPEG}
\gdlproref{READ-PICT}
\gdlfunref{READ-PNG}
\gdlfunref{READ-TIFF}
\gdlfunref{READ-XWD}

\gdlproref{WRITE-BMP}
\gdlproref{WRITE-JPEG}
\gdlproref{WRITE-PICT}
\gdlproref{WRITE-PNG}

\gdlfunref{BYTSCL}
\gdlfunref{CONVOL}
\gdlfunref{MEDIAN}
\gdlfunref{POLY-2D}
\gdlfunref{PREWITT}
\gdlfunref{RADON}
\gdlfunref{ROBERTS}
\gdlfunref{ROTATE}
\gdlfunref{REBIN}
\gdlfunref{SMOOTH}
\gdlfunref{SOBEL}

  
  \chapter{Parallel processing}
  \section{Built-in features (OpenMP)}
  \gdlproref{cpu}

  \section{Semaphores and shared memory (library routines)}
  \gdlfunref{sem-create}
\gdlproref{sem-delete}
\gdlfunref{sem-lock}
\gdlproref{sem-release}

  \section{ImageMagick's features}
  \section{MPI and GDL}

  \chapter{GUI programming (widgets)}
  \gdlfunref{WIDGET-BASE}
\gdlfunref{WIDGET-BUTTON}
\gdlproref{WIDGET-CONTROL}
\gdlfunref{WIDGET-DROPLIST}
\gdlfunref{WIDGET-EVENT}
\gdlfunref{WIDGET-INFO}
\gdlfunref{WIDGET-LABEL}
\gdlfunref{WIDGET-TEXT}

 
  \chapter{Dynamic loading}
  \gdlfunref{CALL-EXTERNAL}
\gdlfunref{LINKIMAGE}


  \chapter{The Python bridge}
  \citet{bib_Python}

  \section{calling Python code from GDL}
  \gdlfunref{PYTHON}
\gdlproref{PYTHON}

  \section{calling GDL code from Python}
  \input{chapters/python-module}

  \onecolumn
  \chapter{Alphabetical list of library routines}
  {
    \secondarysize
    \begin{multicols}{2}{% auto-generated file - do not edit manually!
\gdlroutinedesc{ABS}{1}{1}{}
\gdlroutinedesc{ACOS}{1}{1}{}
\gdlroutinedesc{ALOG}{1}{1}{}
\gdlroutinedesc{ALOG10}{1}{1}{}
\gdlroutinedesc{APPLEMAN}{0}{2}{RESULT}
\gdlroutinedesc{ARG-PRESENT}{1}{1}{}
\gdlroutinedesc{ARRAY-EQUAL}{1}{2}{NO-TYPECONV}
\gdlroutinedesc{ARRAY-INDICES}{1}{2}{}
\gdlroutinedesc{ASIN}{1}{1}{}
\gdlroutinedesc{ASSOC}{1}{3}{PACKED}
\gdlroutinedesc{ATAN}{1}{2}{PHASE}
\gdlroutinedesc{AXIS}{0}{3}{CHARSIZE,CHARTHICK,COLOR,DATA,DEVICE,FONT,NODATA,NOERASE,NORMAL,SUBTITLE,T3D,TICKLEN,XCHARSIZE,XGRIDSTYLE,XMARGIN,XMINOR,XRANGE,XSTYLE,XTHICK,XTICKFORMAT,XTICKLEN,XTICKNAME,XTICKS,XTITLE,YCHARSIZE,YGRIDSTYLE,YMARGIN,YMINOR,YRANGE,YSTYLE,YTHICK,YTICKFORMAT,YTICKLEN,YTICKNAME,YTICKS,YTITLE,ZCHARSIZE,ZGRIDSTYLE,ZMARGIN,ZMINOR,ZRANGE,ZSTYLE,ZTHICK,ZTICKFORMAT,ZTICKLEN,ZTICKNAME,ZTICKS,ZTITLE,ZVALUE,SAVE,XAXIS,YAXIS,XLOG,YLOG,XTYPE,YTYPE,YNOZERO,THICK}
\gdlroutinedesc{BESELI}{1}{2}{DOUBLE,ITER,HELP}
\gdlroutinedesc{BESELJ}{1}{2}{DOUBLE,ITER,HELP}
\gdlroutinedesc{BESELK}{1}{2}{DOUBLE,ITER,HELP}
\gdlroutinedesc{BESELY}{1}{2}{DOUBLE,ITER,HELP}
\gdlroutinedesc{BETA}{1}{2}{DOUBLE}
\gdlroutinedesc{BINDGEN}{1}{8}{}
\gdlroutinedesc{BROYDEN}{1}{2}{DOUBLE,ITMAX,TOLF,TOLX}
\gdlroutinedesc{BYTARR}{1}{8}{NOZERO}
\gdlroutinedesc{BYTE}{1}{10}{}
\gdlroutinedesc{BYTEORDER}{0}{-1}{SSWAP,LSWAP,L64SWAP,SWAP-IF-BIG-ENDIAN,SWAP-IF-LITTLE-ENDIAN,NTOHL,NTOHS,HTONL,HTONS,FTOXDR,DTOXDR,XDRTOF,XDRTOD}
\gdlroutinedesc{BYTSCL}{1}{3}{MIN,MAX,TOP,NAN}
\gdlroutinedesc{CALDAT}{0}{7}{}
\gdlroutinedesc{CALENDAR}{0}{2}{}
\gdlroutinedesc{CALL-EXTERNAL}{1}{-1}{VALUE,ALL-VALUE,RETURN-TYPE,B-VALUE,I-VALUE,L-VALUE,F-VALUE,D-VALUE,UI-VALUE,UL-VALUE,L64-VALUE,UL64-VALUE,S-VALUE,UNLOAD,ALL-GDL,STRUCT-ALIGN-BYTES}
\gdlroutinedesc{CALL-FUNCTION}{1}{-1}{-REF-EXTRA}
\gdlroutinedesc{CALL-METHOD}{0}{-1}{-REF-EXTRA}
\gdlroutinedesc{CALL-METHOD}{1}{-1}{-REF-EXTRA}
\gdlroutinedesc{CALL-PROCEDURE}{0}{-1}{-REF-EXTRA}
\gdlroutinedesc{CATCH}{0}{1}{CANCEL}
\gdlroutinedesc{CD}{0}{1}{CURRENT}
\gdlroutinedesc{CDF-EPOCH}{0}{8}{BREAKDOWN-EPOCH,COMPUTE-EPOCH}
\gdlroutinedesc{CEIL}{1}{1}{L64}
\gdlroutinedesc{CHECK-MATH}{1}{2}{MASK,NOCLEAR,PRINT}
\gdlroutinedesc{CINDGEN}{1}{8}{}
\gdlroutinedesc{CLOSE}{0}{-1}{EXIT-STATUS,FORCE,FILE,ALL}
\gdlroutinedesc{COMMAND-LINE-ARGS}{1}{0}{COUNT}
\gdlroutinedesc{COMPLEX}{1}{10}{}
\gdlroutinedesc{COMPLEXARR}{1}{8}{NOZERO}
\gdlroutinedesc{CONJ}{1}{1}{}
\gdlroutinedesc{CONTOUR}{0}{3}{BACKGROUND,CHARSIZE,CHARTHICK,CLIP,COLOR,DATA,DEVICE,FONT,NOCLIP,NODATA,NOERASE,NORMAL,POSITION,SUBTITLE,T3D,THICK,TICKLEN,TITLE,XCHARSIZE,XGRIDSTYLE,XMARGIN,XMINOR,XRANGE,XSTYLE,XTHICK,XTICKFORMAT,XTICKLEN,XTICKNAME,XTICKS,XTICKV,XTICK-GET,XTITLE,YCHARSIZE,YGRIDSTYLE,YMARGIN,YMINOR,YRANGE,YSTYLE,YTHICK,YTICKFORMAT,YTICKLEN,YTICKNAME,YTICKS,YTICKV,YTICK-GET,YTITLE,ZCHARSIZE,ZGRIDSTYLE,ZMARGIN,ZMINOR,ZRANGE,ZSTYLE,ZTHICK,ZTICKFORMAT,ZTICKLEN,ZTICKNAME,ZTICKS,ZTICKV,ZTICK-GET,ZTITLE,ZVALUE,LEVELS,NLEVELS,MAX-VALUE,MIN-VALUE,AX,AZ,XLOG,XTYPE,YLOG,YTYPE,ZLOG,ZTYPE,FILL,ISOTROPIC,FOLLOW,C-CHARSIZE,OVERPLOT}
\gdlroutinedesc{CONVERT-COORD}{1}{3}{DATA,DEVICE,NORMAL,T3D,DOUBLE,TO-DATA,TO-DEVICE,TO-NORMAL}
\gdlroutinedesc{CONVOL}{1}{3}{CENTER,EDGE-TRUNCATE,EDGE-WRAP}
\gdlroutinedesc{CORRELATE}{1}{2}{DOUBLE,COVARIANCE}
\gdlroutinedesc{COS}{1}{1}{}
\gdlroutinedesc{COSH}{1}{1}{}
\gdlroutinedesc{CPU}{0}{0}{RESET,RESTORE,TPOOL-MAX-ELTS,TPOOL-MIN-ELTS,TPOOL-NTHREADS,VECTOR-ENABLE}
\gdlroutinedesc{CREATE-STRUCT}{1}{-1}{NAME}
\gdlroutinedesc{CROSSP}{1}{2}{}
\gdlroutinedesc{CURSOR}{0}{3}{CHANGE,DOWN,NOWAIT,UP,WAIT,DATA,DEVICE,NORMAL}
\gdlroutinedesc{DBLARR}{1}{8}{NOZERO}
\gdlroutinedesc{DCINDGEN}{1}{8}{}
\gdlroutinedesc{DCOMPLEX}{1}{10}{}
\gdlroutinedesc{DCOMPLEXARR}{1}{8}{NOZERO}
\gdlroutinedesc{DEFSYSV}{0}{3}{EXISTS}
\gdlroutinedesc{DERIV}{1}{2}{NO-CHECK,TEST,HELP}
\gdlroutinedesc{DEVICE}{0}{0}{CLOSE-FILE,FILENAME,LANDSCAPE,PORTRAIT,DECOMPOSED,GET-DECOMPOSED,Z-BUFFERING,SET-RESOLUTION,SET-CHARACTER-SIZE,GET-VISUAL-DEPTH,XSIZE,YSIZE,COLOR,GET-SCREEN-SIZE,INCHES}
\gdlroutinedesc{DIALOG-MESSAGE}{1}{1}{ZENITY-PATH,ZENITY-NAME,HELP,TITLE,RESOURCE-NAME,QUESTION,INFORMATION,ERROR,DISPLAY-NAME,DIALOG-PARENT,DEFAULT-NO,DEFAULT-CANCEL,CENTER,CANCEL}
\gdlroutinedesc{DIALOG-PICKFILE}{1}{0}{VERBOSE,DEBUG,TEST,HELP,ZENITY-PATH,ZENITY-NAME,TITLE,RESOURCE-NAME,WRITE,READ,PATH,OVERWRITE-PROMPT,MUST-EXIST,MULTIPLE-FILES,GROUP,GET-PATH,FIX-FILTER,FILTER,FILE,DISPLAY-NAME,DIALOG-PARENT,DIRECTORY,DEFAULT-EXTENSION}
\gdlroutinedesc{DINDGEN}{1}{8}{}
\gdlroutinedesc{DIST}{1}{2}{}
\gdlroutinedesc{DOUBLE}{1}{10}{}
\gdlroutinedesc{EOF}{1}{1}{}
\gdlroutinedesc{ERASE}{0}{1}{}
\gdlroutinedesc{ERF}{1}{1}{DOUBLE}
\gdlroutinedesc{ERFC}{1}{1}{DOUBLE}
\gdlroutinedesc{ERRORF}{1}{1}{DOUBLE}
\gdlroutinedesc{ESCAPE-SPECIAL-CHAR}{1}{1}{VERBOSE,HELP,TEST,LIST-OF-SPECIAL-CHAR,SHOW-LIST}
\gdlroutinedesc{EXECUTE}{1}{2}{}
\gdlroutinedesc{EXIT}{0}{0}{NO-CONFIRM,STATUS}
\gdlroutinedesc{EXP}{1}{1}{}
\gdlroutinedesc{EXPAND-PATH}{1}{1}{ARRAY,ALL-DIRS,COUNT}
\gdlroutinedesc{EXPINT}{1}{2}{DOUBLE}
\gdlroutinedesc{FACTORIAL}{1}{1}{UL64,STIRLING}
\gdlroutinedesc{FFT}{1}{2}{DOUBLE,INVERSE,OVERWRITE,DIMENSION}
\gdlroutinedesc{FILEPATH}{1}{1}{TMP,TERMINAL,SUBDIRECTORY,ROOT-DIR}
\gdlroutinedesc{FILE-BASENAME}{1}{2}{HELP,FOLD-CASE}
\gdlroutinedesc{FILE-COPY}{0}{2}{TEST,HELP,VERBOSE,QUIET,OVERWRITE,RECURSIVE,NOEXPAND-PATH,ALLOW-SAME,REQUIRE-DIRECTORY}
\gdlroutinedesc{FILE-DELETE}{0}{30}{TEST,HELP,VERBOSE,QUIET,RECURSIVE,NOEXPAND-PATH,ALLOW-NONEXISTENT}
\gdlroutinedesc{FILE-DIRNAME}{1}{1}{HELP,MARK-DIRECTORY}
\gdlroutinedesc{FILE-EXPAND-PATH}{1}{1}{}
\gdlroutinedesc{FILE-INFO}{1}{2}{NOEXPAND-PATH}
\gdlroutinedesc{FILE-LINES}{1}{1}{NOEXPAND-PATH,COMPRESS}
\gdlroutinedesc{FILE-SAME}{1}{2}{NOEXPAND-PATH}
\gdlroutinedesc{FILE-SEARCH}{1}{2}{COUNT,EXPAND-ENVIRONMENT,EXPAND-TILDE,FOLD-CASE,ISSUE-ACCESS-ERROR,MARK-DIRECTORY,NOSORT,QUOTE,MATCH-INITIAL-DOT,MATCH-ALL-INITIAL-DOT,FULLY-QUALIFY-PATH}
\gdlroutinedesc{FILE-TEST}{1}{1}{DIRECTORY,EXECUTABLE,READ,REGULAR,WRITE,ZERO-LENGTH,GET-MODE,BLOCK-SPECIAL,CHARACTER-SPECIAL,NAMED-PIPE,SOCKET,SYMLINK,NOEXPAND-PATH}
\gdlroutinedesc{FINDEX}{1}{2}{}
\gdlroutinedesc{FINDFILE}{1}{1}{HELP,TEST,QUIET,VERBOSE,SPAWN-OPTIONS,SH-LOCATION,COUNT}
\gdlroutinedesc{FINDGEN}{1}{8}{}
\gdlroutinedesc{FINITE}{1}{1}{INFINITY,NAN}
\gdlroutinedesc{FIX}{1}{10}{TYPE,PRINT}
\gdlroutinedesc{FLOAT}{1}{10}{}
\gdlroutinedesc{FLOOR}{1}{1}{L64}
\gdlroutinedesc{FLTARR}{1}{8}{NOZERO}
\gdlroutinedesc{FLUSH}{0}{-1}{}
\gdlroutinedesc{FREE-LUN}{0}{-1}{EXIT-STATUS,FORCE}
\gdlroutinedesc{FSTAT}{1}{1}{}
\gdlroutinedesc{GAMMA}{1}{1}{DOUBLE}
\gdlroutinedesc{GAUSSINT}{1}{1}{DOUBLE}
\gdlroutinedesc{GAUSS-CVF}{1}{1}{}
\gdlroutinedesc{GAUSS-PDF}{1}{1}{}
\gdlroutinedesc{GDL-ERFINV}{1}{1}{DOUBLE}
\gdlroutinedesc{GETENV}{1}{1}{ENVIRONMENT}
\gdlroutinedesc{GET-DRIVE-LIST}{1}{0}{COUNT}
\gdlroutinedesc{GET-KBRD}{1}{1}{}
\gdlroutinedesc{GET-LOGIN-INFO}{1}{0}{}
\gdlroutinedesc{GET-LUN}{0}{1}{}
\gdlroutinedesc{GET-SCREEN-SIZE}{1}{1}{RESOLUTION}
\gdlroutinedesc{GRIBAPI-CLONE}{1}{1}{}
\gdlroutinedesc{GRIBAPI-CLOSE-FILE}{0}{1}{}
\gdlroutinedesc{GRIBAPI-COUNT-IN-FILE}{1}{1}{}
\gdlroutinedesc{GRIBAPI-GET}{0}{3}{}
\gdlroutinedesc{GRIBAPI-GET-DATA}{0}{4}{}
\gdlroutinedesc{GRIBAPI-GET-SIZE}{1}{2}{}
\gdlroutinedesc{GRIBAPI-NEW-FROM-FILE}{1}{1}{}
\gdlroutinedesc{GRIBAPI-OPEN-FILE}{1}{1}{}
\gdlroutinedesc{GRIBAPI-RELEASE}{0}{1}{}
\gdlroutinedesc{GSL-EXP}{1}{1}{}
\gdlroutinedesc{H5A-CLOSE}{0}{1}{}
\gdlroutinedesc{H5A-GET-NAME}{1}{1}{}
\gdlroutinedesc{H5A-GET-NUM-ATTRS}{1}{1}{}
\gdlroutinedesc{H5A-GET-SPACE}{1}{1}{}
\gdlroutinedesc{H5A-GET-TYPE}{1}{1}{}
\gdlroutinedesc{H5A-OPEN-IDX}{1}{2}{}
\gdlroutinedesc{H5A-OPEN-NAME}{1}{2}{}
\gdlroutinedesc{H5A-READ}{1}{1}{}
\gdlroutinedesc{H5D-CLOSE}{0}{1}{}
\gdlroutinedesc{H5D-GET-SPACE}{1}{1}{}
\gdlroutinedesc{H5D-GET-TYPE}{1}{1}{}
\gdlroutinedesc{H5D-OPEN}{1}{2}{}
\gdlroutinedesc{H5D-READ}{1}{1}{}
\gdlroutinedesc{H5F-CLOSE}{0}{1}{}
\gdlroutinedesc{H5F-IS-HDF5}{1}{1}{}
\gdlroutinedesc{H5F-OPEN}{1}{1}{}
\gdlroutinedesc{H5G-CLOSE}{0}{1}{}
\gdlroutinedesc{H5G-OPEN}{1}{2}{}
\gdlroutinedesc{H5S-CLOSE}{0}{1}{}
\gdlroutinedesc{H5S-GET-SIMPLE-EXTENT-DIMS}{1}{1}{}
\gdlroutinedesc{H5T-CLOSE}{0}{1}{}
\gdlroutinedesc{H5T-GET-SIZE}{1}{1}{}
\gdlroutinedesc{H5-GET-LIBVERSION}{1}{0}{}
\gdlroutinedesc{HDF-CLOSE}{0}{1}{}
\gdlroutinedesc{HDF-OPEN}{1}{2}{READ,RDWR,CREATE}
\gdlroutinedesc{HDF-SD-ADDDATA}{0}{2}{START,STRIDE,COUNT}
\gdlroutinedesc{HDF-SD-ATTRFIND}{1}{2}{}
\gdlroutinedesc{HDF-SD-ATTRINFO}{0}{2}{COUNT,DATA,HDF-TYPE,NAME,TYPE}
\gdlroutinedesc{HDF-SD-CREATE}{1}{3}{BYTE,DFNT-INT8,DFNT-UINT8,SHORT,INT,DFNT-INT16,DFNT-UINT16,LONG,DFNT-INT32,DFNT-UINT32,FLOAT,DFNT-FLOAT32,DOUBLE,DFNT-FLOAT64,STRING,DFNT-CHAR,HDF-TYPE}
\gdlroutinedesc{HDF-SD-DIMGET}{0}{1}{NAME,NATTR,SCALE,COUNT}
\gdlroutinedesc{HDF-SD-DIMGETID}{1}{2}{}
\gdlroutinedesc{HDF-SD-END}{0}{1}{}
\gdlroutinedesc{HDF-SD-ENDACCESS}{0}{1}{}
\gdlroutinedesc{HDF-SD-FILEINFO}{0}{3}{}
\gdlroutinedesc{HDF-SD-GETDATA}{0}{2}{START,STRIDE,COUNT}
\gdlroutinedesc{HDF-SD-GETINFO}{0}{1}{DIMS,HDF-TYPE,NAME,NATTS,NDIMS,TYPE,LABEL,UNIT,FORMAT,COORDSYS}
\gdlroutinedesc{HDF-SD-NAMETOINDEX}{1}{2}{}
\gdlroutinedesc{HDF-SD-SELECT}{1}{2}{}
\gdlroutinedesc{HDF-SD-START}{1}{2}{READ,RDWR,CREATE}
\gdlroutinedesc{HDF-VD-ATTACH}{1}{2}{READ,WRITE}
\gdlroutinedesc{HDF-VD-DETACH}{0}{1}{}
\gdlroutinedesc{HDF-VD-FIND}{1}{2}{}
\gdlroutinedesc{HDF-VD-GET}{0}{1}{CLASS,NAME,COUNT,REF,TAG}
\gdlroutinedesc{HDF-VD-READ}{1}{2}{FIELDS,NRECORDS,FULL-INTERLACE,NO-INTERLACE}
\gdlroutinedesc{HDF-VG-ATTACH}{1}{2}{READ,WRITE}
\gdlroutinedesc{HDF-VG-DETACH}{0}{1}{}
\gdlroutinedesc{HDF-VG-GETID}{1}{2}{}
\gdlroutinedesc{HDF-VG-GETINFO}{0}{1}{CLASS,NAME,NENTRIES,REF,TAG}
\gdlroutinedesc{HDF-VG-GETTRS}{0}{3}{}
\gdlroutinedesc{HEAP-GC}{0}{0}{PTR,OBJ,VERBOSE}
\gdlroutinedesc{HELP}{0}{-1}{STRUCTURES,ROUTINES,BRIEF,OUTPUT,PROCEDURES,FUNCTIONS,INFO,LIB,CALLS,RECALL-COMMANDS,MEMORY}
\gdlroutinedesc{HELPFORM}{1}{2}{FULL-STRUCT,TAGFORM,STRUCTURE-NAME,WIDTH,SHORTFORM,SINGLE,SIZE}
\gdlroutinedesc{HISTOGRAM}{1}{1}{BINSIZE,INPUT,MAX,MIN,NBINS,OMAX,OMIN,REVERSE-INDICES,LOCATIONS}
\gdlroutinedesc{HIST-2D}{1}{2}{MIN2,MIN1,MAX2,MAX1,BIN2,BIN1}
\gdlroutinedesc{HIST-ND}{1}{2}{REVERSE-INDICES,NBINS,MAX,MIN}
\gdlroutinedesc{IDENTITY}{1}{1}{DOUBLE}
\gdlroutinedesc{IDL-BASE64}{1}{1}{}
\gdlroutinedesc{IGAMMA}{1}{2}{DOUBLE}
\gdlroutinedesc{IMAGINARY}{1}{1}{}
\gdlroutinedesc{IMSL-BINOMIALCOEF}{1}{2}{DOUBLE}
\gdlroutinedesc{IMSL-CONSTANT}{1}{2}{DOUBLE}
\gdlroutinedesc{IMSL-ERF}{1}{1}{INVERSE,DOUBLE}
\gdlroutinedesc{IMSL-ZEROPOLY}{1}{1}{DOUBLE,COMPANION,JENKINS-TRAUB}
\gdlroutinedesc{IMSL-ZEROSYS}{1}{2}{XGUESS,ITMAX,JACOBIAN,FNORM,ERR-REL,DOUBLE}
\gdlroutinedesc{INDGEN}{1}{8}{TYPE,BYTE,COMPLEX,DCOMPLEX,DOUBLE,FLOAT,L64,LONG,STRING,UINT,UL64,ULONG}
\gdlroutinedesc{INTARR}{1}{8}{NOZERO}
\gdlroutinedesc{INTERPOL}{1}{3}{SPLINE,QUADRATIC,LSQUADRATIC}
\gdlroutinedesc{INTERPOLATE}{1}{4}{CUBIC,GRID}
\gdlroutinedesc{INVERT}{1}{2}{DOUBLE}
\gdlroutinedesc{ISHFT}{1}{2}{-EXTRA}
\gdlroutinedesc{JOURNAL}{0}{1}{}
\gdlroutinedesc{KEYWORD-SET}{1}{1}{}
\gdlroutinedesc{KURTOSIS}{1}{1}{NAN,DOUBLE}
\gdlroutinedesc{L64INDGEN}{1}{8}{}
\gdlroutinedesc{LAGUERRE}{1}{3}{DOUBLE,COEFFICIENTS}
\gdlroutinedesc{LAST-ITEM}{1}{1}{}
\gdlroutinedesc{LA-TRIRED}{0}{3}{DOUBLE,UPPER}
\gdlroutinedesc{LEGENDRE}{1}{3}{DOUBLE}
\gdlroutinedesc{LINDGEN}{1}{8}{}
\gdlroutinedesc{LINKIMAGE}{0}{4}{}
\gdlroutinedesc{LL-ARC-DISTANCE}{1}{3}{DEGREES}
\gdlroutinedesc{LMGR}{1}{0}{SITE-NOTICE,LMHOSTID,INSTALL-NUM,FORCE-DEMO,EXPIRE-DATE,VM,TRIAL,STUDENT,RUNTIME,EMBEDDED,DEMO,CLIENTSERVER}
\gdlroutinedesc{LNGAMMA}{1}{1}{DOUBLE}
\gdlroutinedesc{LOADCT}{0}{1}{SILENT,BOTTOM,NCOLORS,FILE,GET-NAMES}
\gdlroutinedesc{LOADCT-INTERNALGDL}{0}{1}{GET-NAMES}
\gdlroutinedesc{LOCALE-GET}{1}{0}{}
\gdlroutinedesc{LOGICAL-AND}{1}{2}{}
\gdlroutinedesc{LOGICAL-OR}{1}{2}{}
\gdlroutinedesc{LOGICAL-TRUE}{1}{1}{}
\gdlroutinedesc{LON64ARR}{1}{8}{NOZERO}
\gdlroutinedesc{LONARR}{1}{8}{NOZERO}
\gdlroutinedesc{LONG}{1}{10}{}
\gdlroutinedesc{LONG64}{1}{10}{}
\gdlroutinedesc{LUDC}{0}{3}{}
\gdlroutinedesc{MACHAR}{1}{0}{DOUBLE}
\gdlroutinedesc{MAGICK-ADDNOISE}{0}{1}{UNIFORMNOISE,GAUSSIANNOISE,MULTIPLICATIVEGAUSSIANNOISE,IMPULSENOISE,LAPLACIANNOISE,POISSONNOISE,NOISE}
\gdlroutinedesc{MAGICK-CLOSE}{0}{1}{}
\gdlroutinedesc{MAGICK-COLORMAPSIZE}{1}{2}{}
\gdlroutinedesc{MAGICK-COLUMNS}{1}{1}{}
\gdlroutinedesc{MAGICK-CREATE}{1}{3}{}
\gdlroutinedesc{MAGICK-DISPLAY}{0}{1}{}
\gdlroutinedesc{MAGICK-EXISTS}{1}{0}{}
\gdlroutinedesc{MAGICK-FLIP}{0}{1}{}
\gdlroutinedesc{MAGICK-INDEXEDCOLOR}{1}{1}{}
\gdlroutinedesc{MAGICK-INTERLACE}{0}{1}{NOINTERLACE,LINEINTERLACE,PLANEINTERLACE}
\gdlroutinedesc{MAGICK-MAGICK}{1}{2}{}
\gdlroutinedesc{MAGICK-MATTE}{0}{1}{}
\gdlroutinedesc{MAGICK-OPEN}{1}{1}{}
\gdlroutinedesc{MAGICK-PING}{1}{2}{INFO,CHANNELS,DIMENSIONS,HAS-PALETTE,IMAGE-INDEX,NUM-IMAGES,PIXEL-TYPE,TYPE,UNIFORMNOISE,GAUSSIANNOISE,MULTIPLICATIVEGAUSSIANNOISE,IMPULSENOISE,LAPLACIANNOISE,POISSONNOISE,NOISE}
\gdlroutinedesc{MAGICK-QUALITY}{0}{2}{}
\gdlroutinedesc{MAGICK-QUANTIZE}{0}{2}{TRUECOLOR,YUV,GRAYSCALE,DITHER}
\gdlroutinedesc{MAGICK-READ}{1}{1}{RGB,SUB-RECT,MAP}
\gdlroutinedesc{MAGICK-READCOLORMAPRGB}{0}{4}{}
\gdlroutinedesc{MAGICK-READINDEXES}{1}{1}{}
\gdlroutinedesc{MAGICK-ROWS}{1}{1}{}
\gdlroutinedesc{MAGICK-WRITE}{0}{2}{RGB}
\gdlroutinedesc{MAGICK-WRITECOLORTABLE}{0}{4}{}
\gdlroutinedesc{MAGICK-WRITEFILE}{0}{3}{}
\gdlroutinedesc{MAGICK-WRITEINDEXES}{0}{2}{}
\gdlroutinedesc{MAKE-ARRAY}{1}{8}{NOZERO,DIMENSION,INDEX,SIZE,TYPE,VALUE,BYTE,INTEGER,UINT,LONG,ULONG,L64,UL64,FLOAT,DOUBLE,COMPLEX,DCOMPLEX,STRING,PTR,OBJ}
\gdlroutinedesc{MAP-CLIP-SET}{0}{0}{CLIP-UV,TRANSFORM,CLIP-PLANE,SPLIT,RESET}
\gdlroutinedesc{MAP-CONTINENTS}{0}{0}{COLOR,RIVERS,COUNTRIES,HIRES,FILL-CONTINENTS}
\gdlroutinedesc{MAP-PROJ-FORWARD}{1}{3}{CONNECTIVITY,FILL,MAP-STRUCTURE,POLYGONS,POLYLINES,RADIANS}
\gdlroutinedesc{MAP-PROJ-INVERSE}{1}{3}{RADIANS}
\gdlroutinedesc{MATRIX-MULTIPLY}{1}{2}{BTRANSPOSE,ATRANSPOSE}
\gdlroutinedesc{MAX}{1}{2}{MIN,NAN,SUBSCRIPT-MIN,DIMENSION}
\gdlroutinedesc{MEAN}{1}{1}{NAN,DOUBLE}
\gdlroutinedesc{MEANABSDEV}{1}{1}{NAN,DOUBLE}
\gdlroutinedesc{MEDIAN}{1}{2}{EVEN,DOUBLE,DIMENSION}
\gdlroutinedesc{MEMORY}{1}{1}{CURRENT,HIGHWATER,NUM-ALLOC,NUM-FREE,STRUCTURE,L64}
\gdlroutinedesc{MESSAGE}{0}{1}{CONTINUE,INFORMATIONAL,IOERROR,NONAME,NOPREFIX,NOPRINT,RESET,TRACEBACK}
\gdlroutinedesc{MIN}{1}{2}{MAX,NAN,SUBSCRIPT-MAX,DIMENSION}
\gdlroutinedesc{MOMENT}{1}{1}{MAXMOMENT,NAN,DOUBLE,SDEV,MDEV}
\gdlroutinedesc{NCDF-ATTCOPY}{1}{5}{IN-GLOBAL,OUT-GLOBAL}
\gdlroutinedesc{NCDF-ATTDEL}{0}{3}{GLOBAL}
\gdlroutinedesc{NCDF-ATTGET}{0}{4}{GLOBAL}
\gdlroutinedesc{NCDF-ATTINQ}{1}{3}{GLOBAL}
\gdlroutinedesc{NCDF-ATTNAME}{1}{3}{GLOBAL}
\gdlroutinedesc{NCDF-ATTPUT}{0}{4}{GLOBAL,LENGTH,BYTE,CHAR,DOUBLE,FLOAT,LONG,SHORT}
\gdlroutinedesc{NCDF-ATTRENAME}{0}{4}{GLOBAL}
\gdlroutinedesc{NCDF-CLOSE}{0}{1}{}
\gdlroutinedesc{NCDF-CONTROL}{0}{1}{ABORT,ENDEF,FILL,NOFILL,VERBOSE,NOVERBOSE,OLDFILL,REDEF,SYNC}
\gdlroutinedesc{NCDF-CREATE}{1}{1}{CLOBBER,NOCLOBBER}
\gdlroutinedesc{NCDF-DIMDEF}{1}{3}{UNLIMITED}
\gdlroutinedesc{NCDF-DIMID}{1}{2}{}
\gdlroutinedesc{NCDF-DIMINQ}{0}{4}{}
\gdlroutinedesc{NCDF-DIMRENAME}{0}{3}{}
\gdlroutinedesc{NCDF-EXISTS}{1}{0}{}
\gdlroutinedesc{NCDF-INQUIRE}{1}{1}{}
\gdlroutinedesc{NCDF-OPEN}{1}{1}{WRITE,NOWRITE}
\gdlroutinedesc{NCDF-VARDEF}{1}{3}{BYTE,CHAR,DOUBLE,FLOAT,LONG,SHORT}
\gdlroutinedesc{NCDF-VARGET}{0}{3}{COUNT,OFFSET,STRIDE}
\gdlroutinedesc{NCDF-VARGET1}{0}{3}{OFFSET}
\gdlroutinedesc{NCDF-VARID}{1}{2}{}
\gdlroutinedesc{NCDF-VARINQ}{1}{2}{}
\gdlroutinedesc{NCDF-VARPUT}{0}{3}{COUNT,OFFSET,STRIDE}
\gdlroutinedesc{NCDF-VARRENAME}{0}{3}{}
\gdlroutinedesc{NEWTON}{1}{2}{DOUBLE,ITMAX,TOLF,TOLX,HYBRID}
\gdlroutinedesc{NORM}{1}{1}{DOUBLE}
\gdlroutinedesc{N-ELEMENTS}{1}{1}{}
\gdlroutinedesc{N-PARAMS}{1}{1}{}
\gdlroutinedesc{N-TAGS}{1}{1}{DATA-LENGTH,LENGTH}
\gdlroutinedesc{OBJARR}{1}{8}{NOZERO}
\gdlroutinedesc{OBJ-CLASS}{1}{1}{COUNT,SUPERCLASS}
\gdlroutinedesc{OBJ-DESTROY}{0}{-1}{-REF-EXTRA}
\gdlroutinedesc{OBJ-ISA}{1}{2}{}
\gdlroutinedesc{OBJ-NEW}{1}{-1}{-REF-EXTRA}
\gdlroutinedesc{OBJ-VALID}{1}{1}{CAST,COUNT}
\gdlroutinedesc{ON-ERROR}{0}{1}{}
\gdlroutinedesc{OPENR}{0}{3}{APPEND,COMPRESS,BUFSIZE,DELETE,ERROR,F77-UNFORMATTED,GET-LUN,MORE,STDIO,SWAP-ENDIAN,SWAP-IF-BIG-ENDIAN,SWAP-IF-LITTLE-ENDIAN,VAX-FLOAT,WIDTH,XDR,BLOCK,NOAUTOMODE,BINARY,STREAM}
\gdlroutinedesc{OPENU}{0}{3}{APPEND,COMPRESS,BUFSIZE,DELETE,ERROR,F77-UNFORMATTED,GET-LUN,MORE,STDIO,SWAP-ENDIAN,SWAP-IF-BIG-ENDIAN,SWAP-IF-LITTLE-ENDIAN,VAX-FLOAT,WIDTH,XDR,BLOCK,NOAUTOMODE,BINARY,STREAM}
\gdlroutinedesc{OPENW}{0}{3}{APPEND,COMPRESS,BUFSIZE,DELETE,ERROR,F77-UNFORMATTED,GET-LUN,MORE,STDIO,SWAP-ENDIAN,SWAP-IF-BIG-ENDIAN,SWAP-IF-LITTLE-ENDIAN,VAX-FLOAT,WIDTH,XDR,BLOCK,NOAUTOMODE,BINARY,STREAM}
\gdlroutinedesc{OPLOT}{0}{2}{CLIP,COLOR,LINESTYLE,NOCLIP,PSYM,SYMSIZE,T3D,THICK,MAX-VALUE,MIN-VALUE,NSUM,POLAR}
\gdlroutinedesc{PARSE-URL}{1}{1}{}
\gdlroutinedesc{PATH-SEP}{1}{0}{TEST,SEARCH-PATH,PARENT-DIRECTORY}
\gdlroutinedesc{PLOT}{0}{2}{BACKGROUND,CHARSIZE,CHARTHICK,CLIP,COLOR,DATA,DEVICE,FONT,LINESTYLE,NOCLIP,NODATA,NOERASE,NORMAL,POSITION,PSYM,SUBTITLE,SYMSIZE,T3D,THICK,TICKLEN,TITLE,XCHARSIZE,XGRIDSTYLE,XMARGIN,XMINOR,XRANGE,XSTYLE,XTHICK,XTICKFORMAT,XTICKINTERVAL,XTICKLAYOUT,XTICKLEN,XTICKNAME,XTICKS,XTICKUNITS,XTICKV,XTICK-GET,XTITLE,YCHARSIZE,YGRIDSTYLE,YMARGIN,YMINOR,YRANGE,YSTYLE,YTHICK,YTICKFORMAT,YTICKINTERVAL,YTICKLAYOUT,YTICKLEN,YTICKNAME,YTICKS,YTICKUNITS,YTICKV,YTICK-GET,YTITLE,ZCHARSIZE,ZGRIDSTYLE,ZMARGIN,ZMINOR,ZRANGE,ZSTYLE,ZTHICK,ZTICKFORMAT,ZTICKINTERVAL,ZTICKLAYOUT,ZTICKLEN,ZTICKNAME,ZTICKS,ZTICKUNITS,ZTICKV,ZTICK-GET,ZTITLE,ZVALUE,ISOTROPIC,MAX-VALUE,MIN-VALUE,NSUM,POLAR,XLOG,YLOG,YNOZERO,XTYPE,YTYPE}
\gdlroutinedesc{PLOTERR}{0}{4}{TEST,HELP,-EXTRA,BAR-COLOR,LENGTH-OF-HAT,HAT,YLOG,XLOG,YRANGE,XRANGE,TYPE,PSYM}
\gdlroutinedesc{PLOTS}{0}{3}{CLIP,COLOR,LINESTYLE,NOCLIP,PSYM,SYMSIZE,T3D,THICK,DATA,DEVICE,NORMAL}
\gdlroutinedesc{PM}{0}{-1}{FORMAT,TITLE}
\gdlroutinedesc{POINT-LUN}{0}{2}{}
\gdlroutinedesc{POLY}{1}{2}{}
\gdlroutinedesc{POLYFILL}{0}{3}{COLOR,DATA,NORMAL,DEVICE,CLIP,NOCLIP,LINE-FILL,SPACING,LINESTYLE,ORIENTATION,THICK}
\gdlroutinedesc{POLY-2D}{1}{6}{CUBIC,MISSING}
\gdlroutinedesc{POLY-AREA}{1}{2}{SIGNED,DOUBLE}
\gdlroutinedesc{POPD}{0}{0}{}
\gdlroutinedesc{PREWITT}{1}{1}{HELP}
\gdlroutinedesc{PRIMES}{1}{1}{}
\gdlroutinedesc{PRINT}{0}{-1}{FORMAT,AM-PM,DAYS-OF-WEEK,MONTH,STDIO-NON-FINITE}
\gdlroutinedesc{PRINTD}{0}{0}{}
\gdlroutinedesc{PRINTF}{0}{-1}{FORMAT,AM-PM,DAYS-OF-WEEK,MONTH,STDIO-NON-FINITE}
\gdlroutinedesc{PRODUCT}{1}{2}{CUMULATIVE,NAN,INTEGER,PRESERVE-TYPE}
\gdlroutinedesc{PTRARR}{1}{8}{NOZERO,ALLOCATE-HEAP}
\gdlroutinedesc{PTR-FREE}{0}{-1}{}
\gdlroutinedesc{PTR-NEW}{1}{1}{NO-COPY,ALLOCATE-HEAP}
\gdlroutinedesc{PTR-VALID}{1}{1}{CAST,COUNT}
\gdlroutinedesc{PUSHD}{0}{1}{}
\gdlroutinedesc{PYTHON}{0}{-1}{ARGV}
\gdlroutinedesc{PYTHON}{1}{-1}{ARGV,DEFAULTRETURNVALUE}
\gdlroutinedesc{PY-PLOT}{0}{2}{GRID,TITLE,YLABEL,XLABEL}
\gdlroutinedesc{PY-PRINT}{0}{1}{}
\gdlroutinedesc{QUERY-BMP}{1}{2}{}
\gdlroutinedesc{QUERY-DICOM}{1}{2}{}
\gdlroutinedesc{QUERY-GIF}{1}{2}{}
\gdlroutinedesc{QUERY-IMAGE}{1}{2}{-REF-EXTRA}
\gdlroutinedesc{QUERY-JPEG}{1}{2}{}
\gdlroutinedesc{QUERY-PICT}{1}{2}{}
\gdlroutinedesc{QUERY-PNG}{1}{2}{}
\gdlroutinedesc{QUERY-PPM}{1}{2}{}
\gdlroutinedesc{QUERY-TIFF}{1}{2}{IMAGE-INDEX}
\gdlroutinedesc{RADON}{1}{1}{BACKPROJECT,DOUBLE,DRHO,DX,DY,GRAY,LINEAR,NRHO,NTHETA,NX,NY,RHO,RMIN,THETA,XMIN,YMIN}
\gdlroutinedesc{RANDOMN}{1}{8}{DOUBLE,GAMMA,LONG,NORMAL,BINOMIAL,POISSON,UNIFORM}
\gdlroutinedesc{RANDOMU}{1}{8}{DOUBLE,GAMMA,LONG,NORMAL,BINOMIAL,POISSON,UNIFORM}
\gdlroutinedesc{READ}{0}{-1}{FORMAT,AM-PM,DAYS-OF-WEEK,MONTH,PROMPT}
\gdlroutinedesc{READF}{0}{-1}{FORMAT,AM-PM,DAYS-OF-WEEK,MONTH,PROMPT}
\gdlroutinedesc{READS}{0}{-1}{FORMAT,AM-PM,DAYS-OF-WEEK,MONTH}
\gdlroutinedesc{READU}{0}{-1}{TRANSFER-COUNT}
\gdlroutinedesc{READ-ASCII}{1}{1}{VERBOSE,HEADER,TEMPLATE,RECORD-START,NUM-RECORDS,COMMENT-SYMBOL,MISSING-VALUE,DELIMITER,DATA-START,COUNT}
\gdlroutinedesc{READ-BINARY}{1}{1}{ENDIAN,DATA-DIMS,DATA-TYPE,DATA-START,TEMPLATE}
\gdlroutinedesc{READ-BMP}{1}{4}{RGB}
\gdlroutinedesc{READ-DICOM}{1}{4}{IMAGE-INDEX}
\gdlroutinedesc{READ-JPEG}{0}{3}{TWO-PASS-QUANTIZE,TRUE,ORDER,GRAYSCALE,DITHER,COLORS,BUFFER,UNIT}
\gdlroutinedesc{READ-PICT}{0}{5}{}
\gdlroutinedesc{READ-PNG}{1}{4}{TRANSPARENT,VERBOSE,ORDER}
\gdlroutinedesc{READ-TIFF}{1}{4}{VERBOSE,SUB-RECT,PLANARCONFIG,ORIENTATION,INTERLEAVE,IMAGE-INDEX,GEOTIFF,CHANNELS}
\gdlroutinedesc{READ-XWD}{1}{4}{}
\gdlroutinedesc{REAL-PART}{1}{1}{}
\gdlroutinedesc{REBIN}{1}{9}{SAMPLE}
\gdlroutinedesc{RECALL-COMMANDS}{1}{0}{}
\gdlroutinedesc{REFORM}{1}{8}{OVERWRITE}
\gdlroutinedesc{REPLICATE}{1}{9}{}
\gdlroutinedesc{REPLICATE-INPLACE}{0}{6}{}
\gdlroutinedesc{RESOLVE-ROUTINE}{0}{1}{}
\gdlroutinedesc{RESTORE}{0}{1}{RESTORED-OBJECTS,DESCRIPTION,RELAXED-STRUCTURE-ASSIGNMENT,VERBOSE,FILENAME}
\gdlroutinedesc{RETALL}{0}{0}{RETALL}
\gdlroutinedesc{REVERSE}{1}{2}{OVERWRITE}
\gdlroutinedesc{RK4}{1}{5}{DOUBLE,ITER}
\gdlroutinedesc{RK4JMG}{1}{5}{DOUBLE}
\gdlroutinedesc{ROBERTS}{1}{1}{HELP}
\gdlroutinedesc{ROTATE}{1}{2}{}
\gdlroutinedesc{ROUND}{1}{1}{L64}
\gdlroutinedesc{ROUTINE-INFO}{1}{1}{FUNCTIONS,SYSTEM,DISABLED,ENABLED,PARAMETERS}
\gdlroutinedesc{ROUTINE-NAMES}{1}{-1}{LEVEL,VARIABLES,FETCH,ARG-NAME,STORE,S-FUNCTIONS,S-PROCEDURES}
\gdlroutinedesc{RSTRPOS}{1}{3}{}
\gdlroutinedesc{SAVE}{0}{30}{TEST,USEUNIT,NOCATCH,QUIET,ERRMSG,PASS-METHOD,DATA,NAMES,MTIMES,VARSTATUS,STATUS,ALL,APPEND,COMPATIBLE,XDR,VERBOSE,FILENAME}
\gdlroutinedesc{SETENV}{0}{1}{}
\gdlroutinedesc{SET-PLOT}{0}{1}{COPY,INTERPOLATE}
\gdlroutinedesc{SHIFT}{1}{9}{}
\gdlroutinedesc{SHOWFONT}{0}{2}{FIN,BEG,BASE,TT-FONT,ENCAPSULATED}
\gdlroutinedesc{SIN}{1}{1}{}
\gdlroutinedesc{SINDGEN}{1}{8}{}
\gdlroutinedesc{SINH}{1}{1}{}
\gdlroutinedesc{SIZE}{1}{1}{L64,DIMENSIONS,FILE-LUN,N-DIMENSIONS,N-ELEMENTS,STRUCTURE,TNAME,TYPE}
\gdlroutinedesc{SKEWNESS}{1}{1}{NAN,DOUBLE}
\gdlroutinedesc{SKIP-LUN}{0}{2}{TEST,HELP,TRANSFER-COUNT,LINES,EOF}
\gdlroutinedesc{SMOOTH}{1}{2}{VERBOSE,HELP,TEST,NAN,EDGE-TRUNCATE}
\gdlroutinedesc{SOBEL}{1}{1}{HELP}
\gdlroutinedesc{SOCKET}{0}{3}{ERROR,GET-LUN,STDIO,SWAP-ENDIAN,SWAP-IF-BIG-ENDIAN,SWAP-IF-LITTLE-ENDIAN,WIDTH,CONNECT-TIMEOUT,READ-TIMEOUT,WRITE-TIMEOUT}
\gdlroutinedesc{SORT}{1}{1}{L64}
\gdlroutinedesc{SPAWN}{0}{3}{COUNT,EXIT-STATUS,PID,SH,NOSHELL}
\gdlroutinedesc{SPHER-HARM}{1}{4}{DOUBLE}
\gdlroutinedesc{SPL-INIT}{1}{2}{YP0,YPN-1,DOUBLE,HELP}
\gdlroutinedesc{SPL-INIT-OLD}{1}{2}{DEBUG,DOUBLE,YPN-1,YP0}
\gdlroutinedesc{SPL-INTERP}{1}{4}{DOUBLE,HELP}
\gdlroutinedesc{SPL-INTERP-OLD}{1}{4}{DOUBLE}
\gdlroutinedesc{SQRT}{1}{1}{}
\gdlroutinedesc{STDDEV}{1}{1}{NAN,DOUBLE}
\gdlroutinedesc{STOP}{0}{-1}{FORMAT,AM-PM,DAYS-OF-WEEK,MONTH,STDIO-NON-FINITE}
\gdlroutinedesc{STRARR}{1}{8}{NOZERO}
\gdlroutinedesc{STRCMP}{1}{3}{FOLD-CASE}
\gdlroutinedesc{STRCOMPRESS}{1}{1}{REMOVE-ALL}
\gdlroutinedesc{STREGEX}{1}{2}{BOOLEAN,EXTRACT,LENGTH,SUBEXPR,FOLD-CASE}
\gdlroutinedesc{STRING}{1}{-1}{FORMAT,AM-PM,DAYS-OF-WEEK,MONTH,PRINT}
\gdlroutinedesc{STRJOIN}{1}{2}{SINGLE}
\gdlroutinedesc{STRLEN}{1}{1}{}
\gdlroutinedesc{STRLOWCASE}{1}{1}{}
\gdlroutinedesc{STRMID}{1}{3}{REVERSE-OFFSET}
\gdlroutinedesc{STRPOS}{1}{3}{REVERSE-OFFSET,REVERSE-SEARCH}
\gdlroutinedesc{STRPUT}{0}{3}{}
\gdlroutinedesc{STRSPLIT}{1}{2}{HELP,TEST,PRESERVE-NULL,FOLD-CASE,ESCAPE,REGEX,EXTRACT,LENGTH,COUNT}
\gdlroutinedesc{STRTOK}{1}{2}{EXTRACT,ESCAPE,LENGTH,PRESERVE-NULL,REGEX}
\gdlroutinedesc{STRTRIM}{1}{2}{}
\gdlroutinedesc{STRUCT-ASSIGN}{0}{2}{NOZERO,VERBOSE}
\gdlroutinedesc{STRUPCASE}{1}{1}{}
\gdlroutinedesc{STR-SEP}{1}{2}{HELP,TEST,ESC,REMOVE-ALL,TRIM}
\gdlroutinedesc{SURFACE}{0}{3}{BACKGROUND,CHARSIZE,CHARTHICK,CLIP,COLOR,DATA,DEVICE,FONT,LINESTYLE,NOCLIP,NODATA,NOERASE,NORMAL,POSITION,SUBTITLE,T3D,THICK,TICKLEN,TITLE,XCHARSIZE,XGRIDSTYLE,XMARGIN,XMINOR,XRANGE,XSTYLE,XTHICK,XTICKFORMAT,XTICKINTERVAL,XTICKLAYOUT,XTICKLEN,XTICKNAME,XTICKS,XTICKUNITS,XTICKV,XTICK-GET,XTITLE,YCHARSIZE,YGRIDSTYLE,YMARGIN,YMINOR,YRANGE,YSTYLE,YTHICK,YTICKFORMAT,YTICKINTERVAL,YTICKLAYOUT,YTICKLEN,YTICKNAME,YTICKS,YTICKUNITS,YTICKV,YTICK-GET,YTITLE,ZCHARSIZE,ZGRIDSTYLE,ZMARGIN,ZMINOR,ZRANGE,ZSTYLE,ZTHICK,ZTICKFORMAT,ZTICKINTERVAL,ZTICKLAYOUT,ZTICKLEN,ZTICKNAME,ZTICKS,ZTICKUNITS,ZTICKV,ZTICK-GET,ZTITLE,ZVALUE,MAX-VALUE,MIN-VALUE,AX,AZ,XLOG,XTYPE,YLOG,YTYPE,ZLOG,ZTYPE}
\gdlroutinedesc{SVDC}{0}{4}{COLUMN,ITMAX,DOUBLE}
\gdlroutinedesc{SWAP-ENDIAN}{1}{1}{SWAP-IF-LITTLE-ENDIAN,SWAP-IF-BIG-ENDIAN}
\gdlroutinedesc{SWAP-ENDIAN-INPLACE}{0}{1}{SWAP-IF-LITTLE-ENDIAN,SWAP-IF-BIG-ENDIAN}
\gdlroutinedesc{SYSTIME}{1}{2}{JULIAN,SECONDS,UTC}
\gdlroutinedesc{TAG-NAMES}{1}{1}{STRUCTURE-NAME}
\gdlroutinedesc{TAN}{1}{1}{}
\gdlroutinedesc{TANH}{1}{1}{}
\gdlroutinedesc{TEMPLATE}{0}{0}{}
\gdlroutinedesc{TEMPLATE-BLANK}{0}{0}{}
\gdlroutinedesc{TEMPORARY}{1}{1}{}
\gdlroutinedesc{TOTAL}{1}{2}{CUMULATIVE,DOUBLE,NAN,INTEGER,PRESERVE-TYPE}
\gdlroutinedesc{TRACE}{1}{1}{DOUBLE}
\gdlroutinedesc{TRANSPOSE}{1}{2}{}
\gdlroutinedesc{TRIGRID}{1}{6}{MAX-VALUE,MISSING,NX,NY,MAP}
\gdlroutinedesc{TV}{0}{4}{TRUE,NORMAL,CHANNEL,XSIZE,YSIZE,ORDER}
\gdlroutinedesc{TVLCT}{0}{4}{GET,HLS,HSV}
\gdlroutinedesc{TVRD}{1}{5}{CHANNEL,ORDER,TRUE,WORDS}
\gdlroutinedesc{TVSCL}{0}{3}{-EXTRA,NAN}
\gdlroutinedesc{T-PDF}{1}{2}{}
\gdlroutinedesc{UINDGEN}{1}{8}{}
\gdlroutinedesc{UINT}{1}{10}{}
\gdlroutinedesc{UINTARR}{1}{8}{NOZERO}
\gdlroutinedesc{UL64INDGEN}{1}{8}{}
\gdlroutinedesc{ULINDGEN}{1}{8}{}
\gdlroutinedesc{ULON64ARR}{1}{8}{NOZERO}
\gdlroutinedesc{ULONARR}{1}{8}{NOZERO}
\gdlroutinedesc{ULONG}{1}{10}{}
\gdlroutinedesc{ULONG64}{1}{10}{}
\gdlroutinedesc{UNIQ}{1}{2}{}
\gdlroutinedesc{VALUE-LOCATE}{1}{2}{L64}
\gdlroutinedesc{VARIANCE}{1}{1}{NAN,DOUBLE}
\gdlroutinedesc{VOIGT}{1}{2}{DOUBLE,ITER}
\gdlroutinedesc{WAIT}{0}{1}{}
\gdlroutinedesc{WDELETE}{0}{-1}{}
\gdlroutinedesc{WHERE}{1}{2}{COMPLEMENT,NCOMPLEMENT}
\gdlroutinedesc{WIDGET-BASE}{1}{1}{ALIGN-BOTTOM,ALIGN-CENTER,ALIGN-LEFT,ALIGN-RIGHT,ALIGN-TOP,MBAR,MODAL,BASE-ALIGN-BOTTOM,BASE-ALIGN-CENTER,BASE-ALIGN-LEFT,BASE-ALIGN-RIGHT,BASE-ALIGN-TOP,COLUMN,ROW,CONTEXT-EVENTS,CONTEXT-MENU,EVENT-FUNC,EVENT-PRO,EXCLUSIVE,NONEXCLUSIVE,FLOATING,FRAME,FUNC-GET-VALUE,GRID-LAYOUT,GROUP-LEADER,KBRD-FOCUS-EVENTS,KILL-NOTIFY,MAP,NO-COPY,NOTIFY-REALIZE,PRO-SET-VALUE,SCR-XSIZE,SCR-YSIZE,SCROLL,SENSITIVE,SPACE,TITLE,TLB-FRAME-ATTR,TLB-ICONIFY-EVENTS,TLB-KILL-REQUEST-EVENTS,TLB-MOVE-EVENTS,TLB-SIZE-EVENTS,TOOLBAR,TRACKING-EVENTS,UNITS,UNAME,UVALUE,XOFFSET,XPAD,XSIZE,X-SCROLL-SIZE,YOFFSET,YPAD,YSIZE,Y-SCROLL-SIZE,DISPLAY-NAME,RESOURCE-NAME,RNAME-MBAR}
\gdlroutinedesc{WIDGET-BUTTON}{1}{1}{ACCELERATOR,ALIGN-CENTER,ALIGN-LEFT,ALIGN-RIGHT,BITMAP,CHECKED-MENU,DYNAMIC-RESIZE,EVENT-FUNC,EVENT-PRO,FONT,FRAME,FUNC-GET-VALUE,GROUP-LEADER,HELP,KILL-NOTIFY,MENU,NO-COPY,NO-RELEASE,NOTIFY-REALIZE,PRO-SET-VALUE,PUSHBUTTON-EVENTS,SCR-XSIZE,SCR-YSIZE,SENSITIVE,SEPARATOR,TAB-MODE,TOOLTIP,TRACKING-EVENTS,UNAME,UNITS,UVALUE,VALUE,X-BITMAP-EXTRA,XOFFSET,XSIZE,YOFFSET,YSIZE}
\gdlroutinedesc{WIDGET-CONTROL}{0}{1}{REALIZE,MANAGED,EVENT-PRO,XMANAGER-ACTIVE-COMMAND,DESTROY,GET-UVALUE,SET-UVALUE,SET-VALUE,MAP,FUNC-GET-VALUE,PRO-SET-VALUE,SET-UNAME,NO-COPY,SET-BUTTON,SET-DROPLIST-SELECT,SENSITIVE,GET-VALUE}
\gdlroutinedesc{WIDGET-DROPLIST}{1}{1}{DYNAMIC-RESIZE,EVENT-FUNC,EVENT-PRO,FONT,FRAME,FUNC-GET-VALUE,GROUP-LEADER,KILL-NOTIFY,NO-COPY,NOTIFY-REALIZE,PRO-SET-VALUE,RESOURCE-NAME,SCR-XSIZE,SCR-YSIZE,SENSITIVE,TAB-MODE,TITLE,TRACKING-EVENTS,UNAME,UNITS,UVALUE,VALUE,XOFFSET,XSIZE,YOFFSET,YSIZE}
\gdlroutinedesc{WIDGET-EVENT}{1}{1}{XMANAGER-BLOCK,DESTROY}
\gdlroutinedesc{WIDGET-INFO}{1}{1}{VALID,MODAL,MANAGED,XMANAGER-BLOCK,CHILD,VERSION}
\gdlroutinedesc{WIDGET-LABEL}{1}{1}{ALL-EVENTS,CONTEXT-EVENTS,EDITABLE,EVENT-FUNC,EVENT-PRO,FONT,FRAME,FUNC-GET-VALUE,GROUP-LEADER,IGNORE-ACCELERATORS,KBRD-FOCUS-EVENTS,KILL-NOTIFY,NO-COPY,NO-NEWLINE,NOTIFY-REALIZE,PRO-SET-VALUE,RESOURCE-NAME,SCR-XSIZE,SCR-YSIZE,SCROLL,SENSITIVE,TAB-MODE,TRACKING-EVENTS,UNAME,UNITS,UVALUE,VALUE,WRAP,XOFFSET,XSIZE,YOFFSET,YSIZE}
\gdlroutinedesc{WIDGET-TEXT}{1}{1}{ALL-EVENTS,CONTEXT-EVENTS,EDITABLE,EVENT-FUNC,EVENT-PRO,FONT,FRAME,FUNC-GET-VALUE,GROUP-LEADER,IGNORE-ACCELERATORS,KBRD-FOCUS-EVENTS,KILL-NOTIFY,NO-COPY,NO-NEWLINE,NOTIFY-REALIZE,PRO-SET-VALUE,RESOURCE-NAME,SCR-XSIZE,SCR-YSIZE,SCROLL,SENSITIVE,TAB-MODE,TRACKING-EVENTS,UNAME,UNITS,UVALUE,VALUE,WRAP,XOFFSET,XSIZE,YOFFSET,YSIZE}
\gdlroutinedesc{WINDOW}{0}{1}{COLORS,FREE,PIXMAP,RETAIN,TITLE,XPOS,YPOS,XSIZE,YSIZE}
\gdlroutinedesc{WRITEU}{0}{-1}{TRANSFER-COUNT}
\gdlroutinedesc{WRITE-BMP}{0}{5}{RGB,HEADER-DEFINE,IHDR,FOUR-BIT}
\gdlroutinedesc{WRITE-JPEG}{0}{2}{PROGRESSIVE,UNIT,TRUE,QUALITY,ORDER}
\gdlroutinedesc{WRITE-PICT}{0}{5}{}
\gdlroutinedesc{WRITE-PNG}{0}{5}{TRANSPARENT,VERBOSE,ORDER}
\gdlroutinedesc{WSET}{0}{1}{}
\gdlroutinedesc{WSHOW}{0}{2}{}
\gdlroutinedesc{WTN}{1}{2}{COLUMN,DOUBLE,INVERSE,OVERWRITE}
\gdlroutinedesc{XYOUTS}{0}{3}{COLOR,DATA,NORMAL,DEVICE,CLIP,ORIENTATION,ALIGNMENT,CHARSIZE,CHARTHICK,NOCLIP,Z,WIDTH}
}
    \end{multicols}
  }

  \part{Developer's guide}
  \twocolumn

  \chapter{General remarks and coding guidelines}
  ... such as the CERN C++ Coding Standard Specification \citep{bib_CERNcpp} or other similar
  documents.

 
  \chapter{The library-routine API}
  TODO: extract it using Doxygen or some similar tool.


%\begin{lstlisting}[frame=trBL,language=C++] 
%  void AssureDoubleScalarPar( SizeT ix, DDouble& scalar);
%\end{lstlisting}


  \chapter{Extending the documentation}
  \LaTeX

gdldoc.sty

%\BibTeX

Natbib:
%\citet{jon90}	    -->    	Jones et al. [21]
%\citet[chap. 2]{jon90}	    -->    	Jones et al. [21, chap. 2]
%\citep{jon90}	    -->    	[21]
%\citep[chap. 2]{jon90}	    -->    	[21, chap. 2]
%\citep[see][]{jon90}	    -->    	[see 21]
%\citep[see][chap. 2]{jon90}	    -->    	[see 21, chap. 2]
%\citep{jon90a,jon90b}	    -->    	[21, 32]

  \chapter{Extending the testsuite (testsuite/README)}
  \verbatiminput{../../testsuite/README}

  \chapter{A short overview of how GDL works internally}
  Programs (*.pro files) or command line input is parsed (GDLLexer.cpp,
GDLParser.cpp generated with ANTLR from gdlc.g). These results in an
abstract syntax tree (AST) consisting of 'DNode' (dnode.hpp).
This systax tree is further manipulated (compiled) with a tree parser
(GDLTreeParser.cpp generated with ANTLR from gdlc.tree.g,
dcompiler.hpp).
Here the AST is splitted into the different functions/procedures and
the DNode(s) are annotated with further information and converted to
ProgNode(s).
Then these compiled (ProgNode) ASTs are interpreted
(GDLInterpreter.cpp generated with ANTLR from gdlc.i.g, dinterpreter.cpp).


  \chapter{How to make use of OpenMP in GDL}
 
  \chapter{Notes for packagers}
  \section{Optional features of PLplot and ImageMagick}
  \section{The HDF4-netCDF conflict}

  \part{Indices}
  \onecolumn
  \appendix
  \phantomsection
  \secondarysize
  \chapter*{Subject Index}
  \begin{multicols}{3}{\printindex}
  \end{multicols}

  \twocolumn
  \bibliography{gdl}

\end{document}
