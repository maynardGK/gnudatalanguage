GNU Data Language (GDL) is a free/libre/open source incremental compiler 
  compatible with IDL and to some extent with PV-WAVE.
Together with its library routines it serves as a tool for data analysis 
  and visualization in such disciplines as astronomy, geosciences and
  medical imaging. 

GDL as a language is dynamically-typed, vectorized and has 
  object-oriented programming capabilities. 
GDL library routines handle numerical calculations, data visualisation, 
  signal/image processing, interaction with host OS and data input/output. 
GDL supports several data formats such as netCDF, HDF4, HDF5, GRIB, PNG, TIFF, 
  DICOM, etc. 
Graphical output is handled by X11, PostScript, SVG or z-buffer terminals, 
  the last one allowing output graphics (plots) to be saved in a variety of 
  raster graphics formats.
GDL features integrated debugging facilities. 
GDL has also a Python bridge (Python code can be called from GDL; GDL can be compiled 
  as a Python module). 

Packaged versions of GDL are available for several Linux and BSD flavours as well as Mac~OS~X. 
The source code compiles as well on other UNIX systems, including Solaris.
GDL source code is available for download from Sourceforge.net at:
\url{http://sourceforge.net/projects/gnudatalanguage/}.

Other open-source numerical data analysis tools similar to GDL include:
\begin{itemize}
  \item{GNU Octave: \url{http://www.gnu.org/software/octave/}}
  \item{NCL -- NCAR Command Language: \url{http://www.ncl.ucar.edu/}}
  \item{PDL -- Perl Data Language: \url{http://pdl.perl.org/}}
  \item{R: \url{http://www.r-project.org/}}
  \item{Scilab: \url{http://www.scilab.org/}}
  \item{SciPy: \url{http://www.scipy.org/}}
  \item{Yorick: \url{http://yorick.sourceforge.net/}}
\end{itemize}
