This document is divided into two parts:
\begin{itemize}
  \item{User's guide: intended for users developing programs written in GDL,}
  \item{Developer's guide: intended for those interested in developing or packaging GDL.}
\end{itemize}

Most of GDL functionalities are exemplified with short GDL scripts.
For each such script there are two listings provided: a source
  code listing with line numbers to the left and a log of output below, e.g.:
\gdlcodeexample{helloworld_0}{}{}

All scripts are run by invoking \verb=gdl script.pro= what is equivalent
  to loading the script with the @ operator or typing every line of script
  at the GDL's interactive mode command prompt.

Often the scripts contain lines beginning with a dollar sign ''\$'' which is the GDL
  syntax for executing shell commands, e.g.
\gdlcodeexample{helloworld_1}{}{}

If a script involves creation of a plot, the resultant postscript file is
  displayed below the output listing, e.g.:
\gdlcodeexample{helloworld_2}{}{}

While GDL itself reached a beta status of development, the hereby documentation
  is far from reaching an alpha status -- {\bf help is very welcome!}
